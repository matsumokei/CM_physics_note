\documentclass[dvipdfmx,autodetect-engine,12pt]{jsarticle}
\usepackage[utf8]{inputenc}

\usepackage{amsmath,amsfonts,amssymb}
\usepackage{graphicx}
\usepackage[dvipdfmx]{hyperref}
\usepackage{pxjahyper}%these two come together
\usepackage[dvipdfmx]{color}
\usepackage{braket}%dirac notation
\usepackage{wrapfig}
\usepackage{here}
\usepackage{tabularx, dcolumn}
\usepackage{subfigure}
\usepackage{cases}
\usepackage{bigints}%インテグラルで大きくする
\usepackage{mathtools} 
\hypersetup{hidelinks}
\interfootnotelinepenalty=10000 % this is to keep a footnote in a single page
\usepackage{bm}%ベクトル記号
\usepackage{ascmac} %囲い
%%%%%%
\usepackage{tikz}
\usepackage{amsmath}
\usepackage{cases}%連立方程式


%%%%%%newcomand
\newcommand{\be}{\begin{equation}}
\newcommand{\ee}{\end{equation}}
\newcommand{\nn}{\notag \\}

\usepackage{mdframed}%ページをまたぐ    
\newmdenv[skipabove=6mm, skipbelow=4mm]{kotak}


\newtheorem{definition}{定義}[section]
\newtheorem{theorem}[definition]{定理}
\newtheorem{proof}{証明}
\newtheorem{request}[definition]{要請}
\newtheorem{prop}[definition]{命題}
\newtheorem{these}[definition]{仮定}
\newtheorem{lemma}[definition]{補題}
\newtheorem{postlate}[definition]{公理}

%sectionの大きさを変更する
%\usepackage[explicit]{titlesec}

%operator
\newcommand{\hH}{{\hat{H}}}%ハミルトニアン
\newcommand{\hHt}{{\hat{\mathcal{H}}}}%ハミルトニアン
\newcommand{\hU}{{\hat{U}}}
\newcommand{\hM}{{\hat{M}}}
\newcommand{\hN}{{\hat{N}}}
\newcommand{\hA}{{\hat{A}}}
\newcommand{\hB}{{\hat{B}}}
\newcommand{\hO}{{\hat{O}}}
\newcommand{\hAd}{{\hat{A}^\dag}}
\newcommand{\ha}{{\hat{a}}}
\newcommand{\hb}{{\hat{b}}}
\newcommand{\had}{{\hat{a}^\dag}}
\newcommand{\hpsi}{{\hat{\psi}}}
\newcommand{\hpsid}{{\hat{\psi}^\dag}}
\newcommand{\hrho}{{\hat{\rho}}}
\newcommand{\hsig}{{\hat{\sigma}}}
\newcommand{\hx}{{\hat{x}}}
\newcommand{\hy}{{\hat{y}}}
\newcommand{\hz}{{\hat{z}}}
\newcommand{\hX}{{\hat{X}}}
\newcommand{\hY}{{\hat{Y}}}
\newcommand{\hZ}{{\hat{Z}}}
\newcommand{\hp}{{\hat{p}}}
\newcommand{\hvp}{{\hat{\bm p}}}

%%%%%%%%%%%%%%%%%%%%%%%%%%%%%%%%%%%%%%
%%%%%%%%%%%%%%%%%%%%%%%%%%%%%%%%
% USER SPECIFIED COMMANDS
%%%%%%%%%%%%%%%%%%%%%%%%%%%%%%%%
\newcommand{\ie}{i.e.}
\newcommand{\eg}{e.g.}
\newcommand{\etal}{\textit{et al.}}
\newcommand{\e}{\textrm{e}} % Exponential
\newcommand{\hc}{\text{h.c.}} % Hermitian conjugate

\newcommand{\nbar}{\bar{n}}
\newcommand{\adag}{\hat{a}^\dagger}
\newcommand{\adagsq}{\hat{a}^{\dagger 2}}
\newcommand{\hata}{\hat{a}}

\newcommand{\jg}[1]{{\color{orange}#1}}
\newcommand{\dr}[1]{{\color{red}#1}}
\newcommand{\rg}[1]{{\color{MidnightBlue}#1}}
\newcommand{\filler}[1][1]{{\color{gray}\lipsum[1-#1]}}
%%%%%%%%%%%%%%%%%%%%%%%%%%%%%%%%%%%%%

%\vector
\newcommand{\vr}{{\bm{r}}} %vector r
\newcommand{\vP}{{\bm{P}}} %vector r
\newcommand{\vphi}{{\varphi(t,\bm{r})}}

%

\newcommand{\tr}{\mathrm{Tr}}
\newcommand{\diag}{\mathrm{diag}}
\newcommand{\rint}{\mathrm{int}}
\newcommand{\tot}{\mathrm{tot}}


\newcommand{\YM}[1]{\textcolor[rgb]{1, 0.1, 0.1}{#1}}
\newcommand{\YMdel}[1]{\sout{#1}}
%\newcommand{\YMdel}[1]{\textcolor[rgb]{1, 0.1, 0.1}{\sout{\textcolor{black}{#1}}}}
\newcommand{\KM}[1]{\textcolor[rgb]{0.1, 0.1, 1}{#1}}
\newcommand{\KMdel}[1]{\textcolor[rgb]{0.1, 0.1, 0.9}{\sout{\textcolor{black}{#1}}}}


\makeatletter
\title{神経科学}



\begin{document}
\maketitle


\tableofcontents
%目の保護用
%\pagecolor{black}
%\color{white}
%%%%%%%%%%%%%%%%%%%%%

\part{機械学習入門}
\section*{Day1}
\section{線形回帰モデル}
機械学習手法のうち最も基本的なモデルである線形回帰モデルについて述べる.

\subsection{線形回帰モデル}
線形回帰とは,入力$\vec{x}$と出力$y$の間に線形的な関係があると仮定し,訓練データ集合$\mathcal{D}=\{(\vec{x}_1,y_1),\ (\vec{x}_2,y_2),\ (\vec{x}_N,y_N)\}$から,線形モデル
\begin{equation}
    y_i = \vec{w}^{t}\vec{x}_i
    =w_1(\vec{x}_i)_{1}+w_2(\vec{x}_i)_{2}+\cdots+w_d(\vec{x}_i)_{d}
\end{equation}
の未知パラメータ$\vec{w}$を推定し,未知入力に対する予測を行うモデルである.ここで,入力$\vec{x}$は$d$次元ベクトル,出力はスカラーである.式中の表記$(\vec{x}_i)_{j}$は,訓練データ集合$i$番目の入力ベクトルの$j$番目の要素である.これは出力$y_i$を基底$x$で展開し,その展開係数を推定することが目的であるとも言える.


\subsection{特徴量を入れた線形回帰モデル}
入力として,直接訓練集合の入力ベクトル$\vec{x}$を入力するのではなく,入力ベクトル$\vec{x}$を特徴量ベクトル$\vec{\phi}(\vec{x})$へ変換し,特徴量ベクトルに対する線形モデル
\begin{equation}
    y_i = \vec{w}^{t}\vec{\phi}(\vec{x}_i)
\end{equation}
を考えることで,非線形的な関係性を表現することができる.入力ベクトルが非線形変換されているが,パラメータ$\vec{w}$に対して,線形になっていれば,線形回帰モデルである.
\paragraph{例1:}
入力と出力がともにスカラーの場合を考えよう.入力$x$に対する出力$y$の関係を多項式によってフィッティングする場合を考える;
\begin{equation}
    y = \sum_{k=0}^{d} w_k x^{k}
\end{equation}
ここで,$x^{k}$は$x$の$k$乗を表し,$x^0=1$, $w_0=b$である.これは,入力が1次元であったが特徴量ベクトルに変換されたことで,次元が$d$に増えていることに注意せよ.このとき特徴量ベクトルは次のようにかける:
\begin{equation}
     \vec{\phi}(x)=\left(
        \begin{array}{c}
       1\\[10pt]
       \phi_1(x)\\[10pt]
       \phi_2(x)\\[10pt]
       \vdots\\[10pt]
       \phi_d(x)
        \end{array}
        \right)
        =\left(
        \begin{array}{c}
       1\\[10pt]
       x^1\\[10pt]
       x^2\\[10pt]
       \vdots\\[10pt]
       x^d
        \end{array}
        \right)
\end{equation}

訓練データの入力,出力ベクトルの次元がともに$d$次元のときを考える.このとき,入力ベクトルと出力ベクトルはそれぞれ$\vec{x}=(x_1\ x_2\ x_3\ \cdots x_d)$, $\vec{y}=(y_1\ y_2\ y_3\ \cdots y_d)$である.このとき,特徴量ベクトルへの変換は,
\begin{align}
    \vec{y}^{\ t}
    &=\vec{w}^{\ t}\hat{\phi}
    =\vec{w}^{\ t} (\vec{\phi}(x_1)\ \vec{\phi}(x_2)\ \cdots\ \vec{\phi}(x_d))\nn[10pt]
    &=(w_1\ w_2\ \cdots\ w_d)
    \left(
        \begin{array}{cccc}
       \phi_0(x_1)&\phi_0(x_2)&\cdots&\phi_0(x_d)\\[10pt]
       \phi_1(x_1)&\phi_1(x_2)&\cdots&\phi_1(x_d)\\[10pt]
       \phi_2(x_1)&\phi_2(x_2)&\cdots&\phi_2(x_d)\\[10pt]
       \vdots&\vdots&\ddots&\vdots\\[10pt]
       \phi_n(x_1)&\phi_n(x_2)&\cdots&\phi_n(x_d)
        \end{array}
    \right)
\end{align}

\subsection{多次元の線形回帰モデル}
高次元の線形回帰モデル
\begin{equation}
    \vec{y}_i={}^t\vec{w}\vec{x}_i
\end{equation}
を考える.このモデルに対して,訓練データ集合$\mathcal{D}=\{(\vec{x}_1,\vec{y}_1),(\vec{x}_2,\vec{y}_2),\cdots,(\vec{x}_N,\vec{y}_N)\}$が与えられたとき,予測誤差の二乗和
\begin{equation}
    L(\vec{w}) = \sum_{i=1}^{N}\|\vec{y}_i-{}^t\vec{w}\vec{x}_i\|^2
\end{equation}
を損失関数として,損失関数が最小となるようなパラメータ$\vec{w}^{\ast}$を求めれば良い.これは数式で書けば
\begin{equation}
    \vec{w}^{\ast} = \mathop{\arg\min}_{\vec{w}} L(\vec{w})
\end{equation}
である.最小二乗法による最適パラメータ$\vec{w}^{\ast}$は,訓練データ集合から行列
\begin{align}
    \hat{X}
    &
    =\left(
        \begin{array}{c}
       {}^t\vec{x}_1\\[5pt]
       {}^t\vec{x}_2\\[5pt]
       {}^t\vec{x}_3\\[5pt]
       \vdots\\[5pt]
       {}^t\vec{x}_N
        \end{array}
    \right)
    =
    \left(
        \begin{array}{cccc}
       (\vec{x}_1)_1&(\vec{x}_1)_2&\cdots&(\vec{x}_1)_d\\[10pt]
       (\vec{x}_2)_1&(\vec{x}_2)_2&\cdots&(\vec{x}_2)_d\\[10pt]
       (\vec{x}_3)_1&(\vec{x}_3)_2&\cdots&(\vec{x}_3)_d\\[10pt]
       \vdots&\vdots&\ddots&\vdots\\[10pt]
       (\vec{x}_N)_1&(\vec{x}_N)_2&\cdots&(\vec{x}_N)_d\\[10pt]
        \end{array}
    \right),\ \ \ 
    \vec{y}
    &
    =\left(
        \begin{array}{c}
       {y}_1\\[5pt]
       {y}_2\\[5pt]
       {y}_3\\[5pt]
       \vdots\\[5pt]
       {y}_N
        \end{array}
    \right)
\end{align}
を構成したとき,方程式
\begin{equation}
    {}^t\hat{X}\hat{X}\vec{w}^\ast = {}^t\hat{X}\vec{y}
\end{equation}
を満たす.この方程式を正規方程式という.この方程式は${}^t\hat{X}\hat{X}$が正規行列,つまり逆行列が存在すれば$\vec{w}^\ast = ({}^t\hat{X}\hat{X})^{-1}\ {}^t\hat{X}\vec{y}$と解ける.正規方程式の導出を次に示す.
\subsection{具体的な実装・結果}
% \section{教師あり機械学習}
全データを訓練データとテストデータの2種類に分ける.訓練データはトレーニングに用いるデータであり,テストデータは学習し終えたモデルをテストするためのデータである.

用意すべき訓練データとして,入力データ$\mathcal{X}=\{\vec{x}_1,\vec{x}_2,\ldots,\vec{x}_N\}$と出力データ$\mathcal{Y}=\{\vec{y}_1,\vec{y}_2,\ldots,\vec{y}_N\}$を用意する:$\mathcal{D}=\mathcal{X}\times\mathcal{Y}=\{(\vec{x}_1,\vec{y}_1),(\vec{x}_2,\vec{y}_2),\ldots,(\vec{x}_N,\vec{y}_N)\}$ここで,$\vec{x}_i\in\mathbb{R}^n$, $\vec{y}_i\in\mathbb{R}^m$である.機械学習の目的は,与えられた入力$\vec{x}$とその出力$\vec{y}$の関係を知ることである.つまり,
\begin{equation}
    \vec{y} = f(\vec{x})
\end{equation}
という関係において,$f$を知ることが目的である.ここでは,$N$個の訓練データがあるが,その中から1つだけデータを選び,それを$(\vec{x},\vec{y})$とおくこととする.
最も簡単な関数は,次である:
\begin{equation}
    \vec{y}=f(\vec{x}) = \hat{W}\vec{x} + \vec{b}
\end{equation}
\begin{equation}
    y_j= \sum_{i=1}^{n}w_{j,i}x_i + b_i
\end{equation}
ここで,$\vec{y}\in\mathbb{R}^m$, $\vec{x}\in\mathbb{R}^n$, $\hat{W}\in\mathbb{R}^{m\times n}$, $\vec{b}\in\mathbb{R}^m$である:
\begin{equation}
     \vec{y}=\left(
        \begin{array}{c}
       y_1\\[10pt]
       y_2\\[10pt]
       \vdots\\[10pt]
       y_m
        \end{array}
        \right),\ \ \ 
        \vec{x}=\left(
        \begin{array}{c}
          x_1\\[10pt]
       x_2\\[10pt]
       \vdots\\[10pt]
       x_n
        \end{array}
        \right),\ \ \ 
        \vec{b}=\left(
        \begin{array}{c}
          b_1\\[10pt]
       b_2\\[10pt]
       \vdots\\[10pt]
       b_m
        \end{array}
        \right)
\end{equation}

\begin{equation}
     \hat{W}=
     \left(
        \begin{array}{ccccc}
       w_{11}&w_{12}& \dots  & \dots& w_{1n}\\[5pt]
       w_{21}&w_{22}& \dots  & \dots& w_{2n}\\[5pt]
      \vdots&\vdots&\ddots &  & \vdots\\[5pt]
      \vdots&\vdots& &\ddots  & \vdots\\[5pt]
       w_{m1}&w_{m2}& \dots  & \dots& w_{mn}
        \end{array}
        \right)
\end{equation}

次に,次のような関数を考えてみよう.
\begin{equation}
    \vec{y} = \hat{W}_2\{\sigma(\hat{W}_1\vec{x}+\vec{b}_1)\} + \vec{b}_2
\end{equation}
ここで,$\vec{y}\in\mathbb{R}^m$, $\vec{x}\in\mathbb{R}^n$, $\hat{W}_1\in\mathbb{R}^{l\times n}$, $\hat{W}_2\in\mathbb{R}^{m\times l}$, $\vec{b}\in\mathbb{R}^m$, $\vec{b}_1\in\mathbb{R}^l$, $\vec{b}_2\in\mathbb{R}^m$である.また$\sigma(\cdot)$はベクトルに作用し,ベクトルを返す:
\begin{equation}
    \sigma(\vec{x}) = (\sigma(x_1)\ \sigma(x_2)\ \cdots\ \sigma(x_n))
\end{equation}
次に,また$\sigma(\cdot)$を重ねれば,
\begin{equation}
   \vec{y} = \sigma(\hat{W}_3\sigma(\hat{W}_2\sigma(\hat{W}_1\vec{x}+\vec{b}_1) + \vec{b}_2)+\vec{b}_3)
\end{equation}
となる.これは次のように漸化式で書くことができる:
\begin{equation}
    \vec{a}^{l} = \sigma(\hat{W}^{l}\vec{a}^{l-1} + \vec{b}^{l})
\end{equation}
と書ける.ここで,出力と入力(初期値)はそれぞれ$\vec{y}=\vec{a}^{L}$, $\vec{x}=\vec{a}^{0}$とおいた.さらに,非線形関数$\sigma(\cdot)$に通す前の出力を
\begin{equation}
    \vec{z}^{l} = \hat{W}^{l} \vec{a}^{l-1} + \vec{b}^{\ l-1}
\end{equation}
を定義すれば,
\begin{equation}
    \vec{y}=\vec{f}(\vec{x}) = \sigma(\vec{z}^{L}),\ \vec{a}^{0} = \vec{x}
\end{equation}
とまとめることができる.
これを$L$層のニューラルネットワークという.$l$層目の出力$\vec{a}^{l}$の要素$a_j^{l}$の添え字$j$はユニットを表している.

ニューラルネットワーク$f(\cdot)$はある入力$\vec{x}$があったとき,その出力が訓練データ$\vec{y}_i$にできるだけ近くなるように設計された方がよい.つまり,関数としては
\begin{equation}
    \vec{y} = f(\vec{x}) = \sigma(\vec{z}^{l}) \simeq \vec{y}_i
\end{equation}
となるような関数$f$が欲しい.このとき,次のをコスト関数,二乗誤差(Squared error)
\begin{equation}
    E = \|\vec{y}_i- f(\vec{x})\|
\end{equation}
が最小となるように$f$を構築すればよい.ここで,複数の訓練データについては,平均二乗誤差
\begin{equation}
    L(\vec{\theta}) = \frac{1}{N}\sum_{i = 1}^{N}\|\vec{y}_i - f_{\vec{\theta}}(\vec{x}_i)\| 
\end{equation}
をコスト関数として扱う.ここで,$\theta=(\hat{W},\ \vec{b})$であるから,$E$を最小化するための条件は
\begin{align}
    \frac{\partial E}{\partial w_{ij}^{l}}=0,\ \ \ \frac{\partial E}{\partial b_{j}^{l}}=0
\end{align}
であり,この条件を満たす,$\theta=(\hat{W},\ \vec{b})$を決めればよい.このようなパラメータを求めるためには,パラメータに関する微分が重要であることがわかる.





\subsection{}
\begin{equation}
    \hat{C}^{\rm{Ti}}=\Biggl(
    \bm{c}_1^{\rm{Ti}}\ \vec{c}_2^{\rm{Ti}}\ \bm{c}_1^{\rm{Ti}}\ \bm{c}_3^{\rm{Ti}}\ \bm{c}_4^{\rm{Ti}}\ \bm{c}_5^{\rm{Ti}}\ \bm{c}_6^{\rm{Ti}}\ \bm{c}_7^{\rm{Ti}}\ \bm{c}_8^{\rm{Ti}}
    \Biggr)
\end{equation}

\begin{equation}
    \hat{C}^{\rm{O}}
\end{equation}



% \subsection{誤差逆伝播法}
上で述べたニューラルネットワークの最後の$L$層目の出力を具体的に見ると,
\begin{align}
    \vec{y} &= \sigma(\vec{z})^{L}\\[10pt]
    &=\sigma(
        \hat{W}^{(L)}
        \sigma(
            \hat{W}^{(L-1)}
            \sigma(
                \cdots
                \sigma(
                    \hat{W}^{(l+1)}
                    \sigma(
                        \textcolor{red}{\hat{W}^{(l)}}
                        \sigma(
                            \cdots
                        )
                    )
                )
            )
        )
    )
\end{align}
と多重合成関数となっていることがわかる.つまり,層が深くなればなるほど,巨大な合成関数を計算し,その微分を計算する必要がある.そこで,$E$のパラメータ微分を効率よく計算することが誤差逆伝搬法の目的である.

ニューラルネットワークにおいて$l$層目の$j$番目のユニットについて考える.まず,次のデルタを導入する:
\begin{equation}
    \delta_{j}^{l} \equiv 
    \frac{\partial E}{\partial z^{l}_{j}}
\end{equation}
これは,$l$層目における$j$番目の誤差を表す.$\vec{z}^{(l)}$の定義は
\begin{equation}
    \vec{z}^{(l)}
    =\hat{W}^{(l)} \vec{a}^{(l-1)} + \vec{b}^{(l)}
\end{equation}
\begin{equation}
    z^{(l)}_{j}
    =\sum_{k}w^{(l)}_{j,k} a^{(l-1)}_{k} + b^{(l)}_{j}
\end{equation}
であったから,パラメータ$\hat{W}$,$\vec{b}$に関する微分はそれぞれ,
\begin{align}
    \frac{\partial E}{\partial b^{l}_{j}}
    &=\sum_{k}\frac{\partial z^{l}_k}{\partial b^{l}_{j}}
    \frac{\partial E}{\partial z^{l}_{k}}\nn[10pt]
    &=\sum_{k}\delta_{k,j}
    \frac{\partial E}{\partial z^{l}_{k}}
    =\frac{\partial E}{\partial z^{l}_{j}}=\delta^{l}_j,
\end{align}

\begin{align}
    \frac{\partial E}{\partial w^{l}_{i,j}}
    &=\sum_{k}\frac{\partial z^{l}_k}{\partial w^{l}_{i,j}}
    \frac{\partial E}{\partial z^{l}_{k}}\nn[10pt]
    &=\sum_{k}\frac{\partial E}{\partial z^{l}_{k}}
    \delta_{k,i} a^{l-1}_{j}
    =\frac{\partial E}{\partial z^{l}_{i}} a^{l-1}_{j}
    =\delta^{l}_{i}a^{l-1}_{j}
\end{align}
となる.ここで,
\begin{equation}
    z^{l}_{k} = \sum_{s}w^{(l)}_{k,s} a^{(l-1)}_{s} + b^{(l)}_{k}
\end{equation}
であるので,
\begin{equation}
    \frac{\partial z^{l}_k}{\partial w^{l}_{i,j}}
    = \sum_{s}\frac{\partial w^{l}_{k,s}}{\partial w^{l}_{i,j}}
    a^{(l-1)}_{s}
    =\frac{\partial w^{l}_{k,j}}{\partial w^{l}_{i,j}}
    a^{(l-1)}_{j}
    =\delta_{k,i}a^{l-1}_{j}
\end{equation}
が成り立つからこれを用いた.つまり,デルタ$\delta^{l}_{j}$を計算することができれば,パラメータ微分を実行できる.ここで,$a^{l-1}_{j}$は順伝播ですでに計算し保存している値を使えば良い.


まず,$L$層目のデルタ$\delta^{L}_j$について計算を行う.
\begin{align}
    \delta^{L}_{j}
    =\frac{\partial E}{\partial z^{L}_{j}}
    =\sum_{k}\frac{\partial a^{L}_{k}}{\partial z^{L}_{j}}
    \frac{\partial E}{\partial a^{L}_{k}}
\end{align}
ここで,$\vec{a}^{L}=\sigma(\vec{z}^{L})$であるから,
\begin{align}
    \delta^{L}_{j}
    &=\sum_{k}\delta_{k,j}\frac{\partial a^{L}_{k}}{\partial z^{L}_{j}}
    \frac{\partial E}{\partial a^{L}_{k}}\nn[10pt]
    &=\frac{\partial a^{L}_{j}}{\partial z^{L}_{j}}
    \frac{\partial E}{\partial a^{L}_{k}}
    =\frac{\partial \sigma(z^{L}_{j})}{\partial z^{L}_{j}}
    \frac{\partial E}{\partial a^{L}_{k}}
    =\frac{\partial E}{\partial a^{L}_{k}}\sigma^{\prime}(z_j^{L})
\end{align}
と計算できる.関数$\sigma(\cdot)$の微分は解析的に実行可能である.次に,$l$層目でのデルタ$\delta^{l}_{j}$を求める.
\begin{align}
    \delta^{l}_{j}
    =\frac{\partial E}{\partial z^{l}_{j}}
    =\sum_{k}\frac{\partial z^{l+1}_{k}}{\partial z^{l}_{j}}
    \frac{\partial E}{\partial z^{l+1}_{k}}
    =\sum_{k}\frac{\partial z^{l+1}_{k}}{\partial z^{l}_{j}}
    \delta^{l+1}_k
\end{align}
となる.さらに,
\begin{equation}
    \vec{z}^{(l)}
    =\hat{W}^{(l)} \vec{a}^{(l-1)} + \vec{b}^{(l)}
    =\hat{W}^{(l)} \sigma(\vec{z}^{\ l-1}) + \vec{b}^{(l)}
\end{equation}
を使えば,
\begin{equation}
    \frac{\partial z^{l+1}_{k}}{\partial z^{l}_{j}}
    =w^{l+1}_{k,j}\sigma^{\prime}(z_j^{l})
\end{equation}
となり,
\begin{align}
    \delta^{l}_{j}
    &=\sum_{k}w^{l+1}_{k,j}\sigma^{\prime}(z_j^{l})
    \delta^{l+1}_k\nn[10pt]
    &=\sigma^{\prime}(z_j^{l})\sum_{k}w^{l+1}_{k,j}
    \delta^{l+1}_k
\end{align}
を得る.この式から,$l+1$層目のデルタを計算すれば,$l$層目のデルタを計算できることがわかる.ここで,Eq.~\eqref{}の計算は次のように計算できる:
\begin{equation}
    z^{l+1}_{k} = \sum_{s}w^{(l+1)}_{k,s} a^{(l)}_{s} + b^{(l+1)}_{k}
\end{equation}
であるので,
\begin{equation}
    \frac{\partial z^{l+1}_k}{\partial z^{l}_{j}}
    = \sum_{s}w^{(l+1)}_{k,s} 
    \frac{\partial a^{(l)}_{s}}{\partial z^{l}_{j}}
    = \sum_{s}w^{(l+1)}_{k,s} 
    \frac{\partial \sigma(z^{(l)}_{s})}{\partial z^{l}_{j}}
    =w^{l+1}_{k,j}\sigma^{\prime}(z^{l}_j)
\end{equation}

% \section{information theoretic entropy}
\subsection{Shannon entropy}
系が離散的な状態$j=1,\ldots,\Omega$であるとする.このとき,確率分布$\bp=(p_j)_{j=1,\ldots,\Omega}$のShannon entropyは以下で定義される:
\begin{kotak}
	\begin{definition}[Shannon entropy]
	オラクル関数から定まる選択的回転変換$R_f$を任意の$x,y=0,1,\ldots,N-1$に対して
	\be
	S(\bp)\equiv-\sum_{j=1}^{\Omega}p_j\log{p_j}
	\ee
	ここで,$0\log0=0$と定義する.
	\end{definition}
\end{kotak}
確率分布が一様分布
\begin{equation}
    \bp_u= \left(
        \begin{array}{c}
        p_1 \\
        p_2 \\
        \vdots \\
        p_\Omega
        \end{array}
        \right)
        =(p_j)_{j=1,\ldots,\Omega}
\end{equation}
Shannon entopyは
\begin{equation}
    S(\bp_{u})=-\sum_{j=1}^{\Omega}(1/\Omega)\log{(1/\Omega)}
    =\log{\Omega}
\end{equation}
となる.そして,一般的にShannon entropy
\begin{equation}
    0\leq S(\bp)\leq\log{\Omega}
\end{equation}
を満たす.\\
情報量$I_j$は
\begin{equation}
    I_j=\log{1/p_j}
\end{equation}
と書ける.情報量は,$j$を観測したときの,びっくり度合いを表す量であるといえる.Shannon entropyの定義式から,$S(\bp)$は情報量$I_j$の平均であることがわかる.例えば$j$を観測する確率が$p_j=1$のとき,情報量は
\begin{equation}
    I_j=\log{(1/p_j)}=0
\end{equation}
となる.$j$が起こると知っているので,驚きがないことを意味する.次に,$p_j=1/10^{100}$のとき,情報量は
\begin{equation}
    I_j=100\log{10}
\end{equation}
となる.これは,めったに起きないイベントなため,極めて大きな驚きを表すといえる.\\
次に,Shannon entropyの相加性についてみる.系1の確率を$p^{(1)}_j$,系2の確率を$p^{(2)}_k$とし,それぞれ独立とする.この2つの系の複合系を考える.すると,全系の確率は$p_{j,k}p^{(1)}_jp^{(2)}_k$と書ける.全系のShannon entropyは
\begin{align}
    S(\bp)&=-\sum_{j,k}p_{j,k}\log{p_{j,k}}
    =-\sum_j\left(\sum_kp_{j,k}\right)\log{p^{(1)}_j}
    -\sum_k\left(\sum_jp_{j,k}\right)\log{p^{(2)}_k}\nn[10pt]
    &-\sum_jp^{(1)}_j\log{p^{(1)}_j}
    -\sum_kp^{(2)}_k\log{p^{(2)}_k}\nn[10pt]
    &=S(\bp^{(1)})+S(\bp^{(2)})
\end{align}
となり,確かにShannon entropyは相加性を持っていることがわかる.

\subsection{example:2値エントロピー(binary entropy)}
次に重要な例として,2値エントロピーを見る.$\Omega=2$の場合を考え,そのときの確率分布を
\begin{equation}
    \bp_u= \left(
        \begin{array}{c}
        p \\[5pt]
        1-p 
        \end{array}
        \right)
\end{equation}
とすると,Shannon entropyは
\begin{equation}
    S_2(p)\equiv S(\bp)
    =-p\log{p}-(1-p)\log{(1-p)}
\end{equation}
となる.この導関数
\begin{align}
    S_2^{\prime}(p)=log{\frac{1-p}{p}\\[10pt]
    S_2^{\prime\prime}(p)=-\frac{1}{p(1-p)}\\[10pt]
\end{align}
となり,図\ref{}のようになる.さて,エントロピーとは情報量(驚きの量)であった.$p=1/2$のとき,最大値となり,$p=0,1$のとき,ゼロになる.

\subsection{統計力学的エントロピーとのつながり}
カノニカル分布
\begin{equation}
    p^{(can,\beta)}_j=\frac{e^{-\beta E_j}}{Z(\beta)}
\end{equation}
を考える.このとき,Shannon entropyは
\begin{align}
    S(\bp^{(can,\beta)})
    &=-\sum_{j=1}^{\Omega}p^{(can,\beta)}_j\log{\frac{e^{\beta E_j}}{Z(\beta)}}\nn[10pt]
    &=\sum_{j=1}^{\Omega}p^{(can,\beta)}_j
    \{\beta E_j+\log{Z(\beta)}\}
    =\beta\braket{\hH}^{can}_{\beta}+\log{Z(\beta)}\nn[10pt]
    &=\beta\{\braket{\hH}^{can}_{\beta}-F(\beta)\}
    =\frac{1}{T}\{\braket{\hH}^{can}_{\beta}-F(\beta)\}
    =S(\beta)
\end{align}
となる.温度一定の環境下には内部エネルギーと(使うことのできる量)自由エネルギー


\section{相対エントロピーとK-Lダイバージェンス(relative entropy KL divergence)}
\begin{kotak}
	\begin{definition}[relative entropy KL divergence]
	$\bp$,$\bq$を確率分布とする.このとき,相対エントロピー,または,K-Lダイバージェンスは
	\be
	D(\bp|\bq)\equiv\sum_{j=1}^{\Omega}p_j\log{\frac{p_j}{q_j}}
	\ee
	で定義される.このとき,もし,少なくとも1つの$j$に対して,$p_j\neq0$かつ$q_j=0$ならば,$D(\bp|\bq)=\infty$が成り立つ.
	\end{definition}
\end{kotak}
KLダイバージェンスは以下の性質をもつ.
\paragraph{basic property}
非負性
\begin{equation}
    D(\bp|\bq)\geq0
\end{equation}
等号成立条件
\begin{equation}
    D(\bp|\bq)=0\iff \bp=\bq
\end{equation}
これらの証明を行う
\begin{equation}
    \log{x}\leq x-1,\ \ {\rm{for}}\ x>0
\end{equation}
が成り立つことを思い出す.このことから,
\begin{equation}
    \log{\frac{1}{x}}\geq1-x,\ \ ( \log{\frac{1}{x}}>1-x,\ \rm{if}\ x\neq1)
\end{equation}
が成り立つ.このことから,
\begin{align}
    D(\bp|\bq)&=\sum_{j=1}^{\Omega}p_j\log{\frac{1}{q_j/p_j}}\geq \sum_{j=1}^{\Omega}p_j\left\{1-\frac{q_j}{p_j}\right\}=\sum_jp_j-\sum_jq_j=1-1=0
\end{align}
$D(\bp|\bq)$は確率分布$\bp$と$\bq$の間の非対称な距離のことであるといえる.また$D$は$\bq$を基準にしたとき,$\bp$がどれくらい違うか(遠いか)を測っているといえる.例として,一様分布
\begin{equation}
    \bp_u= \left(
        \begin{array}{c}
        p_1 \\
        p_2 \\
        \vdots \\
        p_\Omega
        \end{array}
        \right)
        =(p_j)_{j=1,\ldots,\Omega}
\end{equation}
を基準にKLダイバージェンスを見てみる.$\bq=\bp_u$とすると,
\begin{align}
    D(\bp|\bp_u)=\sum_j p_j(\log{p_j}+\log{\Omega})=\log{\Omega}-S(\bp)
\end{align}
が成り立つ.このことから,Shannon entropyというのは,KLダイバージェンスにおいて,基準を一様分布にしたもの
であるといえる.KLダイバージェンスの非負性から
\begin{equation}
    \log{\Omega}\geq S(\bp)
\end{equation}
であることがわかる.





\subsection{KLダイバージェンスの単調性(monotonicity of KL-divergence)}
$\bp$と$\bq$を任意の確率分布とし,$T$を任意の確率行列とする.このとき,
\begin{equation}
    D(\bp|\bq)\geq D(T\bp|T\bq)
\end{equation}
が成り立つ.つまり,確率分布を時間発展させていくとKL-ダイバージェンスは元のものよりも小さくなることがわかる.\\
証明\\





\section{情報量の推定値ー対数尤度}
真の分布がわかっている場合,相対エントロピー(K-L情報量)によって,モデルが良いか,悪いかを判断できた.しかし,通常は,真の分布は未知であり,真の分布から得られたデータのみが与えられている場合が多い.ここでは,真の分布が未知のときに,どのようにしてこの真の分布を近似するモデルの優劣を比較するかを考える.真の分布$\bp=\{p_1,p_2,\cdots,p_m\}$にしたがって得られた$n$個の観測値$\{x_1,x_2,\ldots,x_n\}$が与えられているとする.各観測値$x_i$は事象$\omega_1,\cdots.\omega_m$のうちどれかひとつを取りうる.各事象$\omega_i$の起きた回数を$n_i$と表すことにすると,$n_1+n_1+\cdots+n_m=n$が成り立つ.ここで,このデータに基づいて,モデル$\bq$に関する相対エントロピーをデータから推定することを考えてみる.K-L情報量は定義から,
\begin{equation}
    D(\bp|\bq)\equiv\sum_{j=1}^{m}p_j\log{\frac{p_j}{q_j}}
    =\sum_{j=1}^{m}p_i\log{p_i}-\sum_{j=1}^{m}p_j\log{q_i}
\end{equation}
と書き換えることができる.右辺初項は,真の分布$\bp$のみに依存した定数である.したがって,右辺債2項が大きいほどK-L情報量$D(\bp|\bq)$は小さくなることがわかる.つまり,K-L情報量の大小を比較するためには,$\sum_{j=1}^{m}p_j\log{q_i}$の値だけが推定できればよい.















\section{}
エントロピー古典的なエントロピーシャノンエントロピー

<h3>
<span id="定義" class="fragment"></span><a href="#%E5%AE%9A%E7%BE%A9"><i class="fa fa-link"></i></a>定義</h3>

 生起確率が$p \space (0 \leq p \leq 1)$である事象Aがあったとします。その事象が実際に生起したときに得られる情報量は、どのように定義されるべきでしょうか?情報の量というからには正の値になっていてほしいです。また、確率が小さい事象が生起したときの方が情報量(=びっくり度=インパクト=得られる知識量)が大きい感覚がありますので、情報量はpについて単調減少とするのが良いです。さらに、2つの事象が同時に発生する確率は両者の積になりますが、そのときに得られる情報量は、各事象が生起したときに得られる情報量の和とする(加法性の要請)のが自然な感覚です<sup id="fnref1"><a href="#fn1" title="もしかすると、ここで「ん?」と思われるかもしれませんが、さらっと流してください。こう考えると万事うまく行くのです。">1</a></sup>。 

 ということを考え合わせ、なるべく簡単な関数形にしたいとすると、情報量を 

-\log p  \tag{1}
 \end{equation}

 のように定義するのが良い、ということがわかります<sup id="fnref2"><a href="#fn2" title="情報理論の習慣に従い、$log$の底は2とし、自然対数(底がe)は$\ln$と書くことにします。">2</a></sup>。 

 情報量を別の言い方で規定することもできます。その事象が生起する前に立ち返ってみると、情報量とは「不確実さの度合い」を表していると言っても良いです。その事象が生起することによって、有限だった不確実さがゼロになり、その分、ゼロだった情報量が有限の値になるという見方です。つまり、「ある事象が生起する前の不確実さの度合い」=「ある事象が生起したときに得られる情報量」という考え方ですね。 

 古典的なエントロピー(シャノン・エントロピーとも呼ばれますが、本記事ではいちいち面倒なので、以後、単にエントロピーと呼ぶことにします)というのは、事象が生起する前の「不確実さの度合い」の平均値(期待値)として定義されます。すなわち、いま、$n$個の事象$\{ A_1,\cdots,A_n\}$があり、その各々の生起確率が$\{ p_1, \cdots , p_n\}$だったとすると、この状況におけるエントロピーは、 

 \begin{equation}H(A) \equiv H(p_1, \cdots, p_n) \equiv -\sum_{i=1}^{n} p_i \log p_i  \tag{2}
 \end{equation}

 と定義されます<sup id="fnref3"><a href="#fn3" title="ここで確率pが0の場合もあり得るので、その場合$p \log p = 0 \log 0 = 0$であると約束しておきます。">3</a></sup>。ここで、 

 \begin{equation}\sum_{i=1}^{n} p_i =1  \tag{3}
 \end{equation}

 です。 

 もう少し説明を加えると、エントロピーというのは、起こり得る事象の系列と各確率がわかっているときに、その状況において定義される指標です。例えば、東京の天気が「晴れ、曇、雨、雪」になるという事象系列があり、各々の確率が「1/2,1/4,1/8,1/8」で与えられたという状況のもとで一意に決まる値です。もし各確率が別の値「1/4,1/4,1/4,1/4」だったとすると、エントロピーは別の値になります。この例の場合、前者と後者を比べると不確実さはどちらが大きいでしょうか?前者の場合、だいたい晴れで、そうでなければ曇りかなーと思っておけば良いのに対し、後者の場合、生起確率が全事象で均一になっているので、どんな天気になるかは全く不明確です。という意味で、不確実さは後者の方が大きいと言えます(つまり、後者の方がエントロピーが大きいです。式(3)に代入して計算してもわかりますが)。そんな感覚を定量化するものであると理解しておけば良いと思います。 

 さらにもう少し説明を加えると、上の例では東京の天気しか考えていませんでしたが、当然別の状況もあり得ます。例えば、大阪の天気が「晴れ、曇、雨、雪」になる可能性と各確率がわかっているという状況とか、あるいは、天気ではなく、いまここにあるサイコロを振ったときの目が「1から6のいずれか」になる可能性と各確率がわかっているといった状況も考えられます。その各々についてエントロピーが定義できます。そのような複数の事象系列の可能性があったときに、それを結合した複合事象についてのエントロピー(結合エントロピー)や、ある事象が生起したことがわかったときに得られるエントロピー(条件付きエントロピー)等々の指標も考えることができます。これらも情報理論的に重要な概念ですが、その話は次節以降で説明することにします。 

 その前に、まず、エントロピーの定義からわかる、いくつかの重要な性質についておさえておきます。 

<h3>
<span id="性質" class="fragment"></span><a href="#%E6%80%A7%E8%B3%AA"><i class="fa fa-link"></i></a>性質</h3>

 エントロピーには、以下の性質があります。 

<ul>
<li>(1) エントロピーの最小値は0(非負)</li>
<li>(2) エントロピーの最大値はlog(n)</li>
<li>(3) エントロピーは凹関数</li>
</ul>

 順に確認します。 

<h4>
<span id="1-エントロピーの最小値は0非負" class="fragment"></span><a href="#1-%E3%82%A8%E3%83%B3%E3%83%88%E3%83%AD%E3%83%94%E3%83%BC%E3%81%AE%E6%9C%80%E5%B0%8F%E5%80%A4%E3%81%AF0%E9%9D%9E%E8%B2%A0"><i class="fa fa-link"></i></a>(1) エントロピーの最小値は0(非負)</h4>

 \begin{equation}H = - \sum_{i=1}^{n} p_i \log p_i \geq 0  \tag{4}
 \end{equation}

 が成り立ちます。$-p_i \log p_i \geq 0$であることから明らかです。等号が成り立つのは、$p_i \log p_i$がすべて0になる場合です。$\sum_i p_i = 1$なので、$\{ p_i \}$のうちでどれか一つが1で、それ以外がすべて0という場合のみ、等号が成り立つことになります。つまり、どれか一つの事象が必ず生起する場合ですね。何が起きるか決まっているので不確実さはゼロ、すなわちエントロピーはゼロになります。 

<h4>
<span id="2-エントロピーの最大値はlogn" class="fragment"></span><a href="#2-%E3%82%A8%E3%83%B3%E3%83%88%E3%83%AD%E3%83%94%E3%83%BC%E3%81%AE%E6%9C%80%E5%A4%A7%E5%80%A4%E3%81%AFlogn"><i class="fa fa-link"></i></a>(2) エントロピーの最大値はlog(n)</h4>

 \begin{equation}H = - \sum_{i=1}^{n} p_i \log p_i \leq \log n  \tag{5}
 \end{equation}

 が成り立ちます。 

 【証明】 

 $\sum_{i=1}^{n} p_i = 1$という制約条件のもとで、式(5)を最大化したいので、ラグランジェの未定乗数法を使います。未定乗数を$\lambda$とし、関数、 

 \begin{equation}f(p_1,\cdots,p_n,\lambda) = - \sum_{i=1}^{n} p_i \log p_i + \lambda (\sum_{i=1}^{n} p_{i} - 1) \tag{6}
 \end{equation}

 を最大化する変数の組$\{ p_i \}$を求めます。そのため、各変数で偏微分したものが0という条件で連立方程式を解きます<sup id="fnref4"><a href="#fn4" title="ラグランジェの未定乗数法では、通常、$\lambda$に関する微分がゼロという条件も使うのですが、それをやらないでも答えが出ることもあります。">4</a></sup>。 

 \begin{equation}\frac{\partial}{\partial p_i} (- \sum_{i}^{n} p_i \log p_i + \lambda (\sum_{i}^{n} p_{i} - 1)) = -\log p_i - \log e +\lambda = 0  \tag{7}
 \end{equation}

 となります。ここで、 

 \begin{equation}\log p = \log e \cdot \ln p  \tag{8}
 \end{equation}

 を使いました。式(7)より、 

\begin{align}
\log p_i & = - \log e + \lambda \\
p_i & = \frac{2^{\lambda}}{e}  \tag{9}
\end{align}


 となり、$p_i$は$i$に依存しない一定値になることがわかります。$\sum_{i=1}^{n} p_i = 1$なので、$p_i = \frac{1}{n}$となり、エントロピーは、 

 \begin{equation}H = n \times \frac{1}{n} (-\log \frac{1}{n}) = \log n  \tag{10}
 \end{equation}

 となります。すべての確率が同じ値、つまり、生起確率に偏りが全くない場合、エントロピーは最大値になります。(証明終) 

 ここで、大事なことを付け加えておきます。それは、「エントロピーはビット数に等しい」ということです。いま、ある事象が確率的に連続して発生するような情報源があったとします。先程は天気の例を出しましたが、今の場合、あまり良い例ではないので、別の例にします。ちょっと抽象的になりますが、{a,b,c,d}という文字が何らかの出現確率で発生する情報源をイメージしてください。これを、デジタル信号にして(つまりビット化して)、ある受信者に間違いなく送信したいとします。さて、何ビット必要でしょうか?という問題を考えます。送信すべき文字が4つとも均等確率で発生するのであれば、容易にわかる通り、2ビット必要ですね。頭の中で$\log 4$を計算したと思います(多分)。文字が256種類あって均等確率で発生するなら、必要なビット数は8ビットですね($\log 256$)。つまり、これは何を計算しているかというと、均等確率の場合のエントロピーを計算していることに相当します。では、出現確率が均等でなかった場合、どうでしょうか。4文字の例の場合で言うと、例えば、aとbの出願確率がとても高く、cやdは滅多に出現しない場合、ビット数は2ビットも必要ないです。極端な話、aとbしか出現しない場合は、1ビットでいけますね。これは、何を計算しているかというと、まさに出現確率が偏っている場合のエントロピーを計算していることになります。つまり、出現確率に応じてそのエントロピーを計算すれば、それが送信に最低限必要なビット数になるということです<sup id="fnref5"><a href="#fn5" title="具体的にどんなビット割当をして符号化(=データ圧縮)すればもっとも効率が良いのかという話題に踏み込むと、本記事がそれで終わってしまうので、感覚的な説明でとりあえずご勘弁ください。">5</a></sup>。出現確率がわからない場合は、エントロピーの最大値である$\log n$に相当するビット数を用意しておけば十分です。 

<h4>
<span id="3-エントロピーは凹関数" class="fragment"></span><a href="#3-%E3%82%A8%E3%83%B3%E3%83%88%E3%83%AD%E3%83%94%E3%83%BC%E3%81%AF%E5%87%B9%E9%96%A2%E6%95%B0"><i class="fa fa-link"></i></a>(3) エントロピーは凹関数</h4>

 確率変数$\{ p_1, \cdots , p_n \}$をn次元ベクトル$p = (p_1, \cdots , p_n)$と見立てると、エントロピー、 

 \begin{equation}H(p) = H(p_1, \cdots , p_n) \tag{11}
 \end{equation}

 は、n次元ベクトルを変数とするスカラー関数とみなすことができます。としたときに、この関数(エントロピー)は、凹関数です。すなわち、任意の実数$x \space (0 \leq x \leq 1)$に対して、 

 \begin{equation}H(xp + (1-x)q) \geq x H(p) + (1-x) H(q)  \tag{12}
 \end{equation}

 が成り立ちます<sup id="fnref6"><a href="#fn6" title="下に凸な関数が凸関数で、下に凹(つまり上に凸)な関数が凹関数です。">6</a></sup>。ここで、$p,q$は確率変数を表すn次元ベクトルです。 

% \part{確率過程}
\section{Introduction}
ここではMarkov過程について考える.Markov過程は主に次の2種類に区別される.
\begin{kotak}
	\begin{definition}[マルコフ連鎖(Markov chain)]
	離散時間における離散状態を取るMarkov過程のこと
	\end{definition}
	\begin{definition}[マルコフジャンプ過程(Markov jump process)]
	連続時間における離散状態を取るMarkov過程のこと
	\end{definition}
\end{kotak}



\section{Markov process with discrete state}

\subsection{Markov chain}
\paragraph{基本的な定義}
とびとびの状態を$j=1,2,\ldots,\Omega$とおく.ある時点で状態が$j$である確率を$p_j$とする.確率$p_j$は規格化条件
\begin{equation}
    0\leq p_j \leq1,\ \ \ \sum_{j=1}^{\Omega}p_j=1
\end{equation}
を満たす.確率$p_j$を並べると,確率分布(確率ベクトル)は,
\begin{equation}
    \bm{p} = \left(
        \begin{array}{c}
        p_1 \\
        p_2 \\
        \vdots \\
        p_\Omega
        \end{array}
        \right)
        =(p_j)_{j=1,\ldots,\Omega}
\end{equation}
と書くことができる.状態が$k$から$j$に移る確率$T_{j,k}$を遷移確率,推移確率 (transition probability)と呼ぶ.そして,遷移確率を成分とする$\Omega\times\Omega$行列
\begin{equation}
    T=(T_{j,k})_{j,k=1,\ldots,\Omega}
\end{equation}
を確率行列と呼ぶ.確率行列の成分である,遷移確率は
\begin{equation}
    0\leq T_{j,k}\leq 1
\end{equation}
と,任意の$k$に対して,規格化条件
\begin{equation}\label{normalize_condition}
    \sum_{j=1}^{\Omega}T_{j,k}=1
\end{equation}
を満たす.
\paragraph{確率の保存則}
任意の確率行列$T$と任意の$\Omega$この成分をもつ列ベクトル$\bm{v}$について,
\begin{equation}
    \sum_{j=1}^{\Omega} (T\bm{v})_i = \sum_{j,k=1}^{\Omega}T_{j,k}(\bm{v})_k = \sum_{j=1}(\bm{v})_j 
\end{equation}
が成り立つ.ここで,2番目の等式で,規格化条件\eqref{normalize_condition}を用いた.





\section{マルコフ連鎖Markov chain}
離散時間$n=0,1,2,\ldots$における確率行列$T^{(n)}$,$n=1,2,\ldots$を考える.\\
基本的なアイデア:時刻$n-1$に系が状態$k$にいたとすると,そのとき,時刻$n$に系が状態$j$にいる確率行列は,$T_{j,k}^{(n)}$となる.\\
確率過程:状態$j$は確率的に決定される.\\
マルコフ過程:直前の時刻の状態が次の状態を決める.\\

時刻$n$に系が状態$j$にいる確率を$p_j^{(n)}$とする.時刻$n$での確率分布を$\bm{p}^{(n)}=(p_j^{(n)})_{j=1,\ldots,\Omega}$とする.時刻$n-1$に系が状態$k$にいる確率に確率行列$T_{j,k}^{(n)}$をかければ,時刻$n$に系が状態$j$にいる確率が得られる:
\begin{equation}
    p_{j}^{(n)}=\sum_{k=1}^{\Omega}T_{j,k}^{(n)}p_k^{(n-1)}
\end{equation}
確率分布で書くと,
\begin{equation}
    \bm{p}^{(n)}=T^{(n)}\bm{p}^{(n-1)}
\end{equation}
である.つまり,時間発展の規則は,確率行列をかければよい.系が初期状態の場合の確率分布を$\bm{p}^{(0)}=(p_1^{(0)},p_2^{(0)},\ldots,p_{\Omega}^{(0)})^{t}$と記述する.すると,時刻1での確率分布は$\bm{p}^{(1)}=T^{1}\bm{p}^{(0)}$であり,以下,$\bm{p}^{(2)}=T^{2}\bm{p}^{(1)}$のように,次々と次の時刻での確率分布が決定される.したがって,
\begin{equation}
    \bm{p}^{(n)}=T^{(n)}T^{(n-1)}\cdots T^{(1)}\bm{p}^{(0}
\end{equation}
と書くことができる.kのように確率分布が時間と共に確率的に変化するのが,有限離散状態・離散時間のマルコフ連鎖である.



\paragraph{単調性}
初期分布は異なるが,状態を遷移させるために用いる確率行列は等しいとする.このとき,
\begin{equation}
    \bm{p}^{(n)}=T^{(n)}\bm{p}^{(n-1)},\ \bm{q}^{(n)}=T^{(n)}\bm{q}^{(n-1)}
\end{equation}
このとき,相対エントロピーは
\begin{equation}
    D(\bm{p}|\bm{q})\geq D(T\bm{p}|T\bm{q})
\end{equation}
が成り立つので,
\begin{equation}
    D(\bm{p}^{(n-1)}|\bm{q}^{(n-1)})\geq D(\bm{p}^{(n)}|\bm{q}^{(n)})
\end{equation}
が成り立つ.つまり,2つの確率分布の距離は決して増えることはないということがわかる.

\paragraph{経路の確率path prob}
初期分布が$\bm{p}^{(0)}$であり,マルコフ連鎖によって時刻$n$まで時間発展したとき,時刻$0,1,\ldots,n$において,系が状態$j_0,j_1,\ldots,j_n$である確率は,
\begin{equation}
    p_{j_0,j_1,\ldots,j_n}\equiv
    T_{j_n,j_{n-1}}^{(n)}T_{j_{n-1},j_{n-2}}^{(n-1)}\cdots T_{j_1,j_0}^{(1)}p_{j_0}^{(0)}
\end{equation}
と定義される.これは,図に示すように,歴史の確率,経路の確率を表しているといえる.
そして,この確率は,
\begin{equation}
    \sum_{j_0,\ldots,j_n=1}^{\Omega}p_{j_0,j_1,\ldots,j_n}=1
\end{equation}
を満たし,確率分布で書けば,
\begin{equation}
    \bm{p}=(p_{j_0,\ldots,j_n})_{j_0,\ldots,j_n=1,\ldots,\Omega}
\end{equation}
$\{1,\ldots,\Omega\}^n$






\subsection{詳細釣り合いの条件(detailed balance condition)}
まず目的分布として,定常分布が与えられとする.この分布をを実現するような遷移確率は何かを考えるという状況を考える.定常分布$\bm{p}^{(s)}=(p_j^{(s)})_{j=1,\ldots,\Omega}$が与えられたとする.すべての$j=1,\ldots,\Omega$に対して,$p^{(s)}_j>0$とする.ここで,詳細つり合いの条件を導入する:

\begin{kotak}
	\begin{definition}[詳細釣り合いの条件(detailed balance condition)]
	もしも,任意の$j\neq k$に対して
	\begin{equation}
	    T_{j,k}\bm{p}^{(s)}_k = T_{k,j}\bm{p}^{(s)}_j
	\end{equation}
	を満たすような確率分布$\bm{p}^{(s)}_j$が存在するとする.このとき,遷移確率$T_{j,k}$ $j,k=1,\ldots,\Omega$は$\bm{p}^{(s)}_j$について詳細釣り合い条件を満たすという.
	\end{definition}
\end{kotak}

詳細釣り合い条件を満たしているとき,
\begin{align}
    \sum_{k=1}^{\Omega} T_{j,k}p_k^{(s)}
    &=T_{j,j}p_k^{(s)} + \sum_{k(\neq j)} T_{j,k}p_k^{(s)}\nn[10pt]
    &=T_{j,j}p_k^{(s)} + \sum_{k(\neq j)} T_{k,j}p_j^{(s)}\nn[10pt]
    &=\left(\sum_{k}^{\Omega} T_{k,j}\right)p_j^{(s)}=p_j^{(s)}
\end{align}
が成り立つ.ここで,2つ目の等号で詳細釣り合い条件を使った.すなわち,
\begin{equation}
    T\bm{p}^{(s)}=\bm{p}^{(s)}
\end{equation}
を得る.逆に$T\bm{p}^{(s)}=\bm{p}^{(s)}$であるから詳細釣り合い条件が成り立つとは限らない.つまり,詳細釣り合い条件は$\bm{p}^{(s)}$が定常分布になるための十分条件であることがわかる.一般に,
\begin{align}
    T\bm{p}^{(s)}&=\bm{p}^{(s)}\\[10pt]
    \sum_{k=1}^{\Omega}T_{k}p^{(s)}_k&=p^{(s)}_k
\end{align}
をつり合い条件 (balanced condition)と呼ぶ.定常分布を用意するための,$T$の決め方はいくらでも考えることができるのだが,その中の一つ (one of them)が詳細つり合い条件であるということに注意が必要である.詳細つり合い条件はとてもシンプルで扱いやすいアイデアなのだが,これを満たすモンテカルロ法は非常に遅いというのが弱点である.次では,目的の分布を用意するための数値計算手法であるマルコフ連鎖モンテカルロ法について解説を行う.また,具体的な実装法についても述べる.



\subsection{マルコフ連鎖モンテカルロ法(Markov chain monte calro method, MCMC)}
\paragraph{メトロポリス法 (metropolis method)}



\section{Markov jump process}
この節では,連続時間における離散状態に関するMarkov過程,すなわちMarkovジャンプ過程について考える.


\subsection{定義}
系は離散状態$j=1,\ldots,\Omega$を取る.系が時刻$t$に状態$j$を取る確率を$p_j(t)$とおく.\footnote{
離散時間の場合は数列として,連続時間の場合は確率分布を時間$t$関数の形として表す.
}
確率$p_j(t)$は任意の時刻$t$で規格化条件を満たすとする
\begin{equation}
    \sum_{j=1}^{\Omega} p_j(t) = 1.
\end{equation}
時間$t$は連続に流れていき,系の状態は,ある瞬間に,ある状態から別の状態へと一瞬でジャンプするとする.このようなジャンプのおこる割合は,過去の記憶に影響されず(Markov性),その瞬間の系の状態だけで決まるとする.

ある時刻に系が状態$j$にいる場合を考える.ここである状態,遷移率(transition probability)と
\begin{kotak}
	\begin{definition}[遷移率 (transition rate) ]
	ある時刻に系が状態$i$にいるとする.それから短い時間間隔$\Delta t$の間に系が別の状態$i$へ遷移している確率を次のように定める.
	\begin{equation}
	    \Delta t\ \omega_{i\to j} + \mathcal{O}((\Delta t)^2)
	\end{equation}
	このとき,単位時間あたりに状態$i$から$j$へ遷移する割合を$\omega_{i\to j}$と書き,これを遷移率(transition rate)と呼ぶ.
	\end{definition}
\end{kotak}
\begin{kotak}
	\begin{definition}[escaoe rate]
	ある時刻に系が状態$j$にいるとする.このとき,状態$j$から$j$以外の別の状態へ逃げていく確率を次のように定義する:
	\begin{equation}
	    \lambda_j(t) = \sum_{k(\neq k)}\omega_{j\to k} \geq 0
	\end{equation}
	これをescape rateと呼ぶ.
	\end{definition}
\end{kotak}

\subsection{Master equation}
状態$j$の$t\sim t+\Delta t$の時間発展を考える.時刻$t+\Delta t$に状態$j$にいる確率は次のように記述される:
\begin{align}
    p_j(t+\Delta t) 
    = -\Biggl\{
    \Delta\ \lambda_j(t) + \mathcal{O}((\Delta t)^2)
    \Biggr\}p_j(t)
    +\sum_{k(\neq j)}
    \Biggl\{
    \Delta\ \omega_{k\to j}(t) + \mathcal{O}((\Delta t)^2)
    \Biggr\}p_k(t)
    +p_j(t)
\end{align}
右辺第一項$-\Bigl\{\Delta\ \lambda_j(t) + \mathcal{O}((\Delta t)^2)\Bigr\}p_j(t)$は状態$j$からescapeする確率を,第二項$\sum_{k(\neq j)}\Bigl\{\Delta\ \omega_{k\to j}(t) + \mathcal{O}((\Delta t)^2)
\Bigr\}p_k(t)$は$k$から$j$に入ってくる確率を,第三項$p_j(t)$は$j$にそのままとどまっている確率を表す.

この式を次のように変形する:
\begin{align}
    p_j(t+\Delta t) - p_j(t)
    = -\Biggl\{
    \Delta\ \lambda_j(t) + \mathcal{O}((\Delta t)^2)
    \Biggr\}p_j(t)
    +\sum_{k(\neq j)}
    \Biggl\{
    \Delta\ \omega_{k\to j}(t) + \mathcal{O}((\Delta t)^2)
    \Biggr\}p_k(t)
\end{align}
そして両辺を$\Delta t$で割り,$\Delta t \to 0$の極限を取ると,次式を得る:
\begin{align}\label{master_equation}
    \frac{d}{dt}p_j(t)
    = -\lambda_j(t)p_j(t)
    +\sum_{k(\neq j)}\omega_{k\to j}(t) p_k(t).
\end{align}

ここで,遷移率行列 (transition rate matrix)を導入する.
\begin{kotak}
	\begin{definition}[遷移率行列 (transition rate matrix)]
	遷移率行列$R(t)=(R_{j,k})_{j,k=1,\ldots,\Omega}$は遷移率$\omega_{k\to j}$とescape rate$\lambda_j$を用いて,次のように定義される:
	\begin{align}
	    R_{j,k}(t) &= \omega_{k\to j}(t) \geq 0\ \ \ (j\neq k)\\[10pt]
	    R_{k,k}(t) &= -\lambda_{k}(t) \leq 0
	\end{align}
	また,遷移率行列は任意の$k$について次を満たす:
	\begin{equation}
	    \sum_{j=1}^{\Omega}R_{j,k} = 0
	\end{equation}
	これはescape rate$\lambda_k(t)$が$\lambda_k(t) = \sum_{j(\neq k)}\omega_{k\to j}(t)$と書けるから,$\sum_{j=1}^{\Omega}R_{j,k} =\sum_{j(\neq k)}R_{j,k} + R_{k,k} = 0$となることからわかる.
	\end{definition}
\end{kotak}
遷移率行列$R(t)$を用いると,微分方程式\eqref{master_equation}は
\begin{equation}
    \frac{d}{dt}p_j(t)
    = \sum_{k = 1}^{\Omega}R_{j,k}\ p_{k}(t)
\end{equation}
あるいは,
\begin{equation}
    \frac{d}{dt}\bm{p}(t)
    = R(t) \bm{p}(t)
\end{equation}
と書ける.この式は物理ではマスター方程式 (master equation)と、数学ではコルモゴロフの先進方程式 (Kolmogorov's forward equation) と呼ぶ.
\subsection{some basic properties}
ここではMarkovジャンプ過程のいくつかの基本的な性質について述べる.
\subsection{Master方程式の行列表現}

\subsection{確率流の計算}

\subsection{convergence theorem for stationary process(定常過程の収束定理)}

\subsection{メトロポリス法}
\part{Boltzmann machine}

線形回帰とは,入力$\vec{x}$と出力$y$の間に線形的な関係があると仮定し,訓練データ集合$\mathcal{D}=\{(\vec{x}_1,y_1),\ (\vec{x}_2,y_2),\ (\vec{x}_N,y_N)\}$
\section{Boltzmann machine}
\section{可視変数のみのボルツマンマシン}
\section{隠れ変数を持つボルツマンマシン}
ボルツマン機械学習では,統計力学で基本となるカノニカル分布 (ボルツマン分布) に従ってデータ$\Vec{x}$が生成されると考える:
\begin{equation}
    P_B(\Vec{x}|\Vec{\Theta}) = \frac{1}{Z_B(\Vec{\Theta})}\exp{-E(\Vec{x}|\Vec{\Theta})}
\end{equation}
ここで,確率分布の引数$\Vec{x}$, $\Vec{\Theta}$はそれぞれイジング変数とパラメータを表す.パラメータ$\Vec{\Theta}$の中身は後で言及を行う.ここで,エネルギー関数は
\begin{equation}
    E(\Vec{x}|\Vec{\gamma},\Vec{c}) = -\sum_{(i,j)\in E}\gamma_{i,j} x_i x_j - \sum_{i \in V}c_ix_i
\end{equation}
のように相互作用項と外場項(バイアス項)の和により定義される.$Z_B(\vec{\Theta})$は規格化定数であり,分配関数と呼ばれ,
次のように定義される:
\begin{equation}
    Z_B(\vec{\Theta}) = \sum_{\Vec{x}\in\{\pm1\}^{n}} \exp{(-E(\Vec{x}|\Vec{\Theta}))}
\end{equation}
ここで,上式中の和の記号は,
\begin{equation}
    \sum_{\Vec{x}} \equiv \prod_{i\in E} \sum_{x_i\in\{\pm 1\}} 
    = \sum_{x_1\in\{\pm 1\}} \sum_{x_2\in\{\pm 1\}} \sum_{x_3\in\{\pm 1\}} 
    \cdots \sum_{x_|E|\in\{\pm 1\}} 
\end{equation}


\begin{align}
     E(\vec{v},\vec{h}) 
    &= -\sum_{i,j}w_{i,j} v_i h_j -\sum_{j,j^\prime} \alpha_{j,j^\prime}h_j h_{j^{\prime}}
    -\sum_{i,i^\prime} \beta_{i,i^\prime} v_i v_{i^{\prime}}\nn[10pt]
    &\hspace{80pt}-\sum_{j} b_{j,j^\prime}h_j  -\sum_{i} a_{i} v_i
\end{align}
これを行列で表すと以下のようになる:
\begin{equation}
        E(\vec{v},\vec{h}) 
        =-\Bigl[
        \Vec{v}^{t}\hat{W}\Vec{h} + \Vec{h}^{t}\hat{A}\Vec{h} + \Vec{v}^{t}\hat{B}\Vec{h}
        + \Vec{a}^{t}\Vec{h}+ \Vec{b}^{t}\Vec{h},
        \Bigr]
\end{equation}
ここで,式中のベクトルと行列はそれぞれである:
\begin{align}
    \hat{A}
    &
    =
    \left(
        \begin{array}{cccc}
       (\vec{x}_1)_1&(\vec{x}_1)_2&\cdots&(\vec{x}_1)_d\\[10pt]
       (\vec{x}_2)_1&(\vec{x}_2)_2&\cdots&(\vec{x}_2)_d\\[10pt]
       (\vec{x}_3)_1&(\vec{x}_3)_2&\cdots&(\vec{x}_3)_d\\[10pt]
       \vdots&\vdots&\ddots&\vdots\\[10pt]
       (\vec{x}_N)_1&(\vec{x}_N)_2&\cdots&(\vec{x}_N)_d\\[10pt]
        \end{array}
    \right),\ \ \ 
    \vec{y}
    &
    =\left(
        \begin{array}{c}
       {y}_1\\[5pt]
       {y}_2\\[5pt]
       {y}_3\\[5pt]
       \vdots\\[5pt]
       {y}_N
        \end{array}
    \right)
\end{align}
\section{RBM}
制限ボルツマンマシンは
\begin{equation}
    \alpha_{j,j^\prime} = \beta_{i,i^\prime} = 0
\end{equation}
すなわち,
\begin{equation}
    \Vec{a}^{t}\Vec{h} = \Vec{b}^{t}\Vec{h} = 0
\end{equation}
としたものをいう.これは,可視層同士,隠れ層同士の結合を考えないモデルに帰着する.制限ボルツマンマシンのエネルギー関数は

\begin{align}
     E(\vec{v},\vec{h}) 
    &= -\sum_{i,j}w_{i,j} v_i h_j -\sum_{j} b_{j}h_j  -\sum_{i} a_{i} v_i\\[10pt]
    &=-\Vec{v}^{t}\hat{W}\Vec{h} - \Vec{h}^{t}\hat{A}\Vec{h} - \Vec{v}^{t}\hat{B}\Vec{h}
\end{align}
これを行列で表すと以下のようになる:
\subsection{制限ボルツマンマシンの条件付き確率の独立性}
可視層を固定したもとでの,制限ボルツマンマシンの隠れ層の条件付き確率は,以下のように書ける:
\begin{equation}
    P(\vec{h}|\vec{v};\vec{\Theta}) 
    = \frac{P(\vec{h},\vec{v};\vec{\Theta})}{P(\vec{v}|\vec{\Theta})}
    =\prod_{i=1}^{N} \frac{\exp{\lambda^{H}_j h_j}}{2\cosh{\lambda^{H}_j h_j}}
\end{equation}
ここで,
\begin{equation}
    \lambda^{H}_i \equiv b_i + \sum_{j=1}^{N} w_{i,j} v_j.
\end{equation}
また,隠れ層を固定したもとでの,制限ボルツマンマシンの可視層の条件付き確率は,以下のように書ける:
\begin{equation}
    P(\vec{v}|\vec{h};\vec{\Theta}) 
    = \frac{P(\vec{v},\vec{h};\vec{\Theta})}{P(\vec{h}|\vec{\Theta})}
    =\prod_{i=1}^{N} \frac{\exp{\lambda^{V}_j v_j}}{2\cosh{\lambda^{V}_j v_j}}
\end{equation}
ここで,
\begin{equation}
    \lambda^{V}_i \equiv a_i + \sum_{j=1}^{N} w_{i,j} h_j.
\end{equation}

これを証明する:
\begin{equation}
    P(\vec{v}|\vec{h};\vec{\Theta}) = \frac{P(\vec{v},\vec{h};\vec{\Theta})}{P(\vec{h}|\vec{\Theta})}
\end{equation}
を考える.可視変数$\vec{v}$に関する周辺確率は
\begin{equation}
    P(\vec{h}|\vec{\Theta}) = \sum_{\vec{v}}P(\vec{v},\vec{h};\vec{\Theta})
\end{equation}
と書ける.これを用いることで,
\begin{align}
    P(\vec{v}|\vec{h};\vec{\Theta}) 
    &= \frac{P(\vec{v},\vec{h};\vec{\Theta})}{P(\vec{h}|\vec{\Theta})}
    = \frac{P(\vec{v},\vec{h};\vec{\Theta})}{\sum_{\vec{v}}P(\vec{v},\vec{h};\vec{\Theta})}\nn[10pt]
    &=\frac{\exp{\biggl[\sum_{i,j}w_{i,j} v_i h_j  +\sum_{i} a_{i} v_i
    \textcolor{blue}{+\sum_{j} b_{j}h_j}
    \biggr]}}
    {\sum_{\vec{v}}\exp{\biggl[\sum_{i,j}w_{i,j} v_i h_j  +\sum_{i} a_{i} v_i
    \textcolor{blue}{+\sum_{j} b_{j}h_j} 
    \biggr]}}\nn[10pt]
    &=\frac{\exp{\biggl[\sum_{i}v_i\Bigl\{\sum_{i,j}w_{j}  h_j  + a_{i}\Bigr\}
    \textcolor{blue}{+\sum_{j} b_{j}h_j}
    \biggr]}}
    {\sum_{\vec{v}}\exp{\biggl[\sum_{i}v_i\Bigl\{\sum_{j}w_{i,j}  h_j  + a_{i}\Bigr\}
    \textcolor{blue}{+\sum_{j} b_{j}h_j}
    \biggr]}}
\end{align}
ここで,$\vec{v}$に対する和に関係のない項$\textcolor{blue}{+\sum_{j} b_{j}h_j}$は約分できる:
\begin{align}
    P(\vec{v}|\vec{h};\vec{\Theta}) 
    &=\frac{\exp{\biggl[\sum_{i}v_i\Bigl\{\sum_{j}w_{i,j}  h_j  + a_{i}\Bigr\}\biggr]
    \exp\biggl[\textcolor{blue}{+\sum_{j} b_{j}h_j}
    \biggr]}}
    {\sum_{\vec{v}}\exp{\biggl[\sum_{i}v_i\Bigl\{\sum_{i,j}w_{j}  h_j  + a_{i}\Bigr\}\biggr]
    \exp\biggl[\textcolor{blue}{+\sum_{j} b_{j}h_j}
    \biggr]}}\nn[10pt]
    &=\frac{\exp{\biggl[\sum_{i}v_i\Bigl\{\sum_{j}w_{i,j}  h_j  + a_{i}\Bigr\}
    \biggr]}}
    {\sum_{\vec{v}}\exp{\biggl[\sum_{i}v_i\Bigl\{\sum_{j}w_{i,j}  h_j  + a_{i}\Bigr\}
    \biggr]}}\nn[10pt]
    &=\frac{\prod_{i=1}^{N}\exp{\biggl[v_i\Bigl\{\sum_{j}w_{i,j}  h_j  + a_{i}\Bigr\}
    \biggr]}}
    {\sum_{\vec{v}}\prod_{i=1}^{N}\exp{\biggl[v_i\Bigl\{\sum_{j}w_{i,j}  h_j  + a_{i}\Bigr\}
    \biggr]}}\nn[10pt]
    &=\frac{\prod_{i=1}^{N}\exp{\biggl[v_i\Bigl\{\sum_{j}w_{i,j}  h_j  + a_{i}\Bigr\}
    \biggr]}}
    {\sum_{\vec{v}}\prod_{i=1}^{N}\exp{\biggl[v_i\Bigl\{\sum_{j}w_{i,j}  h_j  + a_{i}\Bigr\}
    \biggr]}}
\end{align}
ここで,
\begin{align}
    \sum_{\vec{v}}\prod_{i=1}^{N}\exp{\biggl[v_i\Bigl\{\sum_{j}w_{i,j}  h_j  + a_{i}\Bigr\}
    \biggr]}
    &=\prod_{i=1}^{N}\biggl[
    \sum_{v_i}\exp{\biggl[v_i\Bigl\{\sum_{j}w_{i,j}  h_j  + a_{i}\Bigr\}
    \biggr]
    \biggr]}\nn[10pt]
    &=\prod_{i=1}^{N}
    2\cosh{\biggl[v_i\Bigl\{\sum_{j}w_{i,j}  h_j  + a_{i}\Bigr\}
    \biggr]}
\end{align}


\begin{align}
    P(\vec{v}|\vec{h};\vec{\Theta}) 
    &=\prod_{i=1}^{N}
    \frac{\exp{\biggl[v_i\Bigl\{\sum_{j}w_{i,j}  h_j  + a_{i}\Bigr\}
    \biggr]}}
    {
    2\cosh{\biggl[v_i\Bigl\{\sum_{j}w_{i,j}  h_j  + a_{i}\Bigr\}
    \biggr]}}
\end{align}




\section{neural quantum state}

\section{重要な公式集}
\begin{equation}
    \sum_{x=\pm1}\exp{(ax)}=\exp{a}+\exp{-a} = 2\cosh{a}
\end{equation}
多変数への拡張
\begin{align}
    \sum_{\vec{x}\in\{\pm1\}^{N}}\exp{(\vec{a}^{\ t}\vec{x})}
    &=\sum_{\vec{x}\in\{\pm1\}^{N}}\exp{\Bigl(\sum_{i=1}^N a_ix_i\Bigr)}\nn[10pt]
    &=\sum_{\vec{x}\in\{\pm1\}^{N}}\prod_{i=1}^N\exp{\Bigl( a_ix_i\Bigr)}\nn[10pt]
    &=\prod_{i=1}^N\Biggl[
    \sum_{x_i=\{\pm1\}}\exp{\Bigl( a_ix_i\Bigr)}
    \Biggr]\nn[10pt]
    &=\prod_{i=1}^N\Biggl[
    2\cosh{(a_ix_i)}
    \Biggr]
\end{align}
ここで,2番目から3番目の等式に移る際に,
\begin{equation}
    \sum_{x_1=\pm1}\sum_{x_2=\pm1}\cdots\sum_{x_N=\pm1}
    \prod_{i=1}^N\exp{\Bigl( a_ix_i\Bigr)}\nn[10pt]
    =\prod_{i=1}^N\Biggl[
    \sum_{x_i=\{\pm1\}}\exp{\Bigl( a_ix_i\Bigr)}
    \Biggr]
\end{equation}
という関係式が一般的に成り立つことを用いた.これは,2次元の場合に簡単に確認できる:
\begin{align}
    \sum_{x_1=\pm1}\sum_{x_2=\pm1}
    \prod_{i=1}^2\exp{\Bigl( a_ix_i\Bigr)}
    &=\sum_{x_1=\pm1}\sum_{x_2=\pm1}
    \exp{\Bigl( a_1 x_1 + a_2 x_2\Bigr)}\nn[10pt]
    &=
    e^{a_1 + a_2}
    +e^{a_1  - a_2 }
    +e^{- a_1  + a_2 }
    +e^{- a_1 - a_2}\nn[10pt]
    &=
    e^{a_1}
    (e^{a_2}+e^{ - a_2 })
    +e^{- a_1 }
    (e^{a_2}+e^{ - a_2 })\nn[10pt]
    &=
    (e^{a_1}+e^{- a_1 })
    (e^{a_2}+e^{ - a_2 })\nn[10pt]
    &=\prod_{i=1}^2\Biggl[
    \sum_{x_i=\{\pm1\}}\exp{\Bigl( a_ix_i\Bigr)}
    \Biggr]
\end{align}




\section{量子ボルツマン機械学習}

% 
\section{最適化の手法}



\section{生成モデル}


\section{オートエンコーダー}

\section{変分オートエンコーダー}
\section*{Introduction: Generative Models}
\addcontentsline{toc}{section}{\protect\numberline{}Introduction: Generative Models}%

Given observed samples $\bm{x}$ from a distribution of interest, the goal of a \textbf{generative model} is to learn to \textit{model} its true data distribution $p(\bm{x})$.  Once learned, we can \textit{generate} new samples from our approximate model at will.  Furthermore, under some formulations, we are able to use the learned model to evaluate the likelihood of observed or sampled data as well.

There are several well-known directions in current literature, that we will only introduce briefly at a high level.  Generative Adversarial Networks (GANs) model the sampling procedure of a complex distribution, which is learned in an adversarial manner.  Another class of generative models, termed "likelihood-based", seeks to learn a model that assigns a high likelihood to the observed data samples.  This includes autoregressive models, normalizing flows, and Variational Autoencoders (VAEs).  Another similar approach is energy-based modeling, in which a distribution is learned as an arbitrarily flexible energy function that is then normalized.  Score-based generative models are highly related; instead of learning to model the energy function itself, they learn the \textit{score} of the energy-based model as a neural network.  In this work we explore and review diffusion models, which as we will demonstrate, have both likelihood-based and score-based interpretations.  We showcase the math behind such models in excruciating detail, with the aim that anyone can follow along and understand what diffusion models are and how they work.

\section*{Background: ELBO, VAE, and Hierarchical VAE}
\addcontentsline{toc}{section}{\protect\numberline{}Background: ELBO, VAE, and Hierarchical VAE}%

For many modalities, we can think of the data we observe as represented or generated by an associated unseen \textit{latent} variable, which we can denote by random variable $\bm{z}$.  The best intuition for expressing this idea is through Plato's \href{https://en.wikipedia.org/wiki/Allegory_of_the_cave}{Allegory of the Cave}.  In the allegory, a group of people are chained inside a cave their entire life and can only see the two-dimensional shadows projected onto a wall in front of them, which are generated by unseen three-dimensional objects passed before a fire.  To such people, everything they observe is actually determined by higher-dimensional abstract concepts that they can never behold.

Analogously, the objects that we encounter in the actual world may also be generated as a function of some higher-level representations; for example, such representations may encapsulate abstract properties such as color, size, shape, and more.  Then, what we observe can be interpreted as a three-dimensional projection or instantiation of such abstract concepts, just as what the cave people observe is actually a two-dimensional projection of three-dimensional objects.  Whereas the cave people can never see (or even fully comprehend) the hidden objects, they can still reason and draw inferences about them; in a similar way, we can approximate latent representations that describe the data we observe.

Whereas Plato’s Allegory illustrates the idea behind latent variables as potentially unobservable representations that determine observations, a caveat of this analogy is that in generative modeling, we generally seek to learn lower-dimensional latent representations rather than higher-dimensional ones.  This is because trying to learn a representation of higher dimension than the observation is a fruitless endeavor without strong priors.  On the other hand, learning lower-dimensional latents can also be seen as a form of compression, and can potentially uncover semantically meaningful structure describing observations.


\subsubsection*{Evidence Lower Bound}
\addcontentsline{toc}{section}{\protect\numberline{}\protect\numberline{}Evidence Lower Bound}%

Mathematically, we can imagine the latent variables and the data we observe as modeled by a joint distribution $p(\bm{x}, \bm{z})$.  Recall one approach of generative modeling, termed "likelihood-based", is to learn a model to maximize the likelihood $p(\bm{x})$ of all observed $\bm{x}$.  There are two ways we can manipulate this joint distribution to recover the likelihood of purely our observed data $p(\bm{x})$; we can explicitly \href{https://en.wikipedia.org/wiki/Marginal_likelihood}{marginalize} out the latent variable $\bm{z}$:
\begin{equation}
\label{eq:1}
p(\bm{x}) = \int p(\bm{x}, \bm{z})d\bm{z}
\end{equation}
or, we could also appeal to the \href{https://en.wikipedia.org/wiki/Chain_rule_(probability)}{chain rule of probability}:
\begin{equation}
\label{eq:2}
p(\bm{x}) = \frac{p(\bm{x}, \bm{z})}{p(\bm{z}|\bm{x})}
\end{equation}
Directly computing and maximizing the likelihood $p(\bm{x})$ is difficult because it either involves integrating out all latent variables $\bm{z}$ in Equation \ref{eq:1}, which is intractable for complex models, or it involves having access to a ground truth latent encoder $p(\bm{z}|\bm{x})$ in Equation \ref{eq:2}.  However, using these two equations, we can derive a term called the \textbf{E}vidence \textbf{L}ower \textbf{Bo}und (ELBO), which as its name suggests, is a \href{https://en.wikipedia.org/wiki/Upper_and_lower_bounds}{lower bound} of the evidence.  The evidence is quantified in this case as the log likelihood of the observed data.  Then, maximizing the ELBO becomes a proxy objective with which to optimize a latent variable model; in the best case, when the ELBO is powerfully parameterized and perfectly optimized, it becomes exactly equivalent to the evidence.  Formally, the equation of the ELBO is:
\begin{equation}
\mathbb{E}_{q_{\bm{\phi}}(\bm{z}|\bm{x})}\left[\log\frac{p(\bm{x}, \bm{z})}{q_{\bm{\phi}}(\bm{z}|\bm{x})}\right]
\end{equation}
To make the relationship with the evidence explicit, we can mathematically write:
\begin{equation}
\log p(\bm{x}) \geq \mathbb{E}_{q_{\bm{\phi}}(\bm{z}|\bm{x})}\left[\log\frac{p(\bm{x}, \bm{z})}{q_{\bm{\phi}}(\bm{z}|\bm{x})}\right]
\end{equation}
Here, $q_{\bm{\phi}}(\bm{z}|\bm{x})$ is a flexible approximate variational distribution with parameters $\bm{\phi}$ that we seek to optimize.  Intuitively, it can be thought of as a parameterizable model that is learned to estimate the true distribution over latent variables for given observations $\bm{x}$; in other words, it seeks to approximate true posterior $p(\bm{z}|\bm{x})$.  As we will see when exploring the Variational Autoencoder, as we increase the lower bound by tuning the parameters $\bm{\phi}$ to maximize the ELBO, we gain access to components that can be used to model the true data distribution and sample from it, thus learning a generative model.  For now, let us try to dive deeper into why the ELBO is an objective we would like to maximize.

Let us begin by deriving the ELBO, using Equation \ref{eq:1}:
\begin{align}
\log p(\bm{x}) & = \log \int p(\bm{x}, \bm{z})d\bm{z} && \text{(Apply Equation \ref{eq:1})}\\
           & = \log \int \frac{p(\bm{x}, \bm{z})q_{\bm{\phi}}(\bm{z}|\bm{x})}{q_{\bm{\phi}}(\bm{z}|\bm{x})}d\bm{z} && \text{(Multiply by $1 = \frac{q_{\bm{\phi}}(\bm{z}|\bm{x})}{q_{\bm{\phi}}(\bm{z}|\bm{x})}$)}\\
           & = \log \mathbb{E}_{q_{\bm{\phi}}(\bm{z}|\bm{x})}\left[\frac{p(\bm{x}, \bm{z})}{q_{\bm{\phi}}(\bm{z}|\bm{x})}\right] && \text{(Definition of Expectation)}\\
           & \geq \mathbb{E}_{q_{\bm{\phi}}(\bm{z}|\bm{x})}\left[\log \frac{p(\bm{x}, \bm{z})}{q_{\bm{\phi}}(\bm{z}|\bm{x})}\right] && \text{(Apply \href{https://en.wikipedia.org/wiki/Jensen\%27s_inequality}{Jensen's Inequality})} \label{eq:8}
\end{align}
In this derivation, we directly arrive at our lower bound by applying Jensen's Inequality.  However, this does not supply us much useful information about what is actually going on underneath the hood; crucially, this proof gives no intuition on exactly why the ELBO is actually a lower bound of the evidence, as Jensen's Inequality handwaves it away.  Furthermore, simply knowing that the ELBO is truly a lower bound of the data does not really tell us why we want to maximize it as an objective.  To better understand the relationship between the evidence and the ELBO, let us perform another derivation, this time using Equation \ref{eq:2}:
\begin{align}
\log p(\bm{x}) & = \log p(\bm{x}) \int q_{\bm{\phi}}(\bm{z}|\bm{x})dz && \text{(Multiply by $1 = \int q_{\bm{\phi}}(\bm{z}|\bm{x})d\bm{z}$)}\\
          & = \int q_{\bm{\phi}}(\bm{z}|\bm{x})(\log p(\bm{x}))dz && \text{(Bring evidence into integral)}\\
          & = \mathbb{E}_{q_{\bm{\phi}}(\bm{z}|\bm{x})}\left[\log p(\bm{x})\right] && \text{(Definition of Expectation)}\\
          & = \mathbb{E}_{q_{\bm{\phi}}(\bm{z}|\bm{x})}\left[\log\frac{p(\bm{x}, \bm{z})}{p(\bm{z}|\bm{x})}\right]&& \text{(Apply Equation \ref{eq:2})}\\
          & = \mathbb{E}_{q_{\bm{\phi}}(\bm{z}|\bm{x})}\left[\log\frac{p(\bm{x}, \bm{z})q_{\bm{\phi}}(\bm{z}|\bm{x})}{p(\bm{z}|\bm{x})q_{\bm{\phi}}(\bm{z}|\bm{x})}\right]&& \text{(Multiply by $1 = \frac{q_{\bm{\phi}}(\bm{z}|\bm{x})}{q_{\bm{\phi}}(\bm{z}|\bm{x})}$)}\\
          & = \mathbb{E}_{q_{\bm{\phi}}(\bm{z}|\bm{x})}\left[\log\frac{p(\bm{x}, \bm{z})}{q_{\bm{\phi}}(\bm{z}|\bm{x})}\right] + \mathbb{E}_{q_{\bm{\phi}}(\bm{z}|\bm{x})}\left[\log\frac{q_{\bm{\phi}}(\bm{z}|\bm{x})}{p(\bm{z}|\bm{x})}\right] && \text{(Split the Expectation)}\\
          & = \mathbb{E}_{q_{\bm{\phi}}(\bm{z}|\bm{x})}\left[\log\frac{p(\bm{x}, \bm{z})}{q_{\bm{\phi}}(\bm{z}|\bm{x})}\right] + \infdiv{q_{\bm{\phi}}(\bm{z}|\bm{x})}{p(\bm{z}|\bm{x})}  && \text{(Definition of \href{https://en.wikipedia.org/wiki/Kullback\%E2\%80\%93Leibler_divergence}{KL Divergence})}\label{eq:15}\\
          & \geq \mathbb{E}_{q_{\bm{\phi}}(\bm{z}|\bm{x})}\left[\log\frac{p(\bm{x}, \bm{z})}{q_{\bm{\phi}}(\bm{z}|\bm{x})}\right]  && \text{(KL Divergence always $\geq 0$)}
\end{align}
From this derivation, we clearly observe from Equation \ref{eq:15} that the evidence is equal to the ELBO plus the KL Divergence between the approximate posterior $q_{\bm{\phi}}(\bm{z}|\bm{x})$ and the true posterior $p(\bm{z}|\bm{x})$.  In fact, it was this KL Divergence term that was magically removed by Jensen's Inequality in Equation \ref{eq:8} of the first derivation.  Understanding this term is the key to understanding not only the relationship between the ELBO and the evidence, but also the reason why optimizing the ELBO is an appropriate objective at all.

Firstly, we now know why the ELBO is indeed a lower bound: the difference between the evidence and the ELBO is a strictly non-negative KL term, thus the value of the ELBO can never exceed the evidence.

Secondly, we explore why we seek to maximize the ELBO.  Having introduced latent variables $\bm{z}$ that we would like to model, our goal is to learn this underlying latent structure that describes our observed data.  In other words, we want to optimize the parameters of our variational posterior $q_{\bm{\phi}}(\bm{z}|\bm{x})$ to exactly match the true posterior distribution $p(\bm{z}|\bm{x})$, which is achieved by minimizing their KL Divergence (ideally to zero).  Unfortunately, it is intractable to minimize this KL Divergence term directly, as we do not have access to the ground truth $p(\bm{z}|\bm{x})$ distribution.  However, notice that on the left hand side of Equation \ref{eq:15}, the likelihood of our data (and therefore our evidence term $\log p(\bm{x})$) is always a constant with respect to $\bm{\phi}$, as it is computed by marginalizing out all latents $\bm{z}$ from the joint distribution $p(\bm{x}, \bm{z})$ and does not depend on $\bm{\phi}$ whatsoever.  Since the ELBO and KL Divergence terms sum up to a constant, any maximization of the ELBO term with respect to $\bm{\phi}$ necessarily invokes an equal minimization of the KL Divergence term.  Thus, the ELBO can be maximized as a proxy for learning how to perfectly model the true latent posterior distribution; the more we optimize the ELBO, the closer our approximate posterior gets to the true posterior.  Additionally, once trained, the ELBO can be used to estimate the likelihood of observed or generated data as well, since it is learned to approximate the model evidence $\log p(\bm{x})$.

\subsubsection*{Variational Autoencoders}
\addcontentsline{toc}{section}{\protect\numberline{}\protect\numberline{}Variational Autoencoders}%
\begin{figure}
  \centering
  %includegraphics[width=0.25\linewidth]{images/vae.png}
  \caption{A Variational Autoencoder graphically represented.  Here, encoder $q(\bm{z}|\bm{x})$ defines a distribution over latent variables $\bm{z}$ for observations $\bm{x}$, and $p(\bm{x}|\bm{z})$ decodes latent variables into observations.}
  \label{fig:vae}
\end{figure}

In the default formulation of the Variational Autoencoder (VAE)~\cite{kingma2013auto}, we directly maximize the ELBO.  This approach is \textit{variational}, because we optimize for the best $q_{\bm{\phi}}(\bm{z}|\bm{x})$ amongst a family of potential posterior distributions parameterized by $\bm{\phi}$.  It is called an \textit{autoencoder} because it is reminiscent of a traditional autoencoder model, where input data is trained to predict itself after undergoing an intermediate bottlenecking representation step.  To make this connection explicit, let us dissect the ELBO term further:
\begin{align}
\scalemath{0.98}{\mathbb{E}_{q_{\bm{\phi}}(\bm{z}|\bm{x})}\left[\log\frac{p(\bm{x}, \bm{z})}{q_{\bm{\phi}}(\bm{z}|\bm{x})}\right]}
&= \scalemath{0.98}{\mathbb{E}_{q_{\bm{\phi}}(\bm{z}|\bm{x})}\left[\log\frac{p_{\bm{\theta}}(\bm{x}|\bm{z})p(\bm{z})}{q_{\bm{\phi}}(\bm{z}|\bm{x})}\right]}         && \scalemath{0.98}{\text{(Chain Rule of Probability)}}\\
&= \scalemath{0.98}{\mathbb{E}_{q_{\bm{\phi}}(\bm{z}|\bm{x})}\left[\log p_{\bm{\theta}}(\bm{x}|\bm{z})\right] + \mathbb{E}_{q_{\bm{\phi}}(\bm{z}|\bm{x})}\left[\log\frac{p(\bm{z})}{q_{\bm{\phi}}(\bm{z}|\bm{x})}\right]}         && \scalemath{0.98}{\text{(Split the Expectation)}}\\
&= \underbrace{\scalemath{0.98}{\mathbb{E}_{q_{\bm{\phi}}(\bm{z}|\bm{x})}\left[\log p_{\bm{\theta}}(\bm{x}|\bm{z})\right]}}_\text{reconstruction term} - \underbrace{\scalemath{0.98}{\infdiv{q_{\bm{\phi}}(\bm{z}|\bm{x})}{p(\bm{z})}}}_\text{prior matching term} && \scalemath{0.98}{\text{(Definition of KL Divergence)}}\label{eq:19}
\end{align}
In this case, we learn an intermediate bottlenecking distribution $q_{\bm{\phi}}(\bm{z}|\bm{x})$ that can be treated as an \textit{encoder}; it transforms inputs into a distribution over possible latents.  Simultaneously, we learn a deterministic function $p_{\bm{\theta}}(\bm{x}|\bm{z})$ to convert a given latent vector $\bm{z}$ into an observation $\bm{x}$, which can be interpreted as a \textit{decoder}.

The two terms in Equation \ref{eq:19} each have intuitive descriptions: the first term measures the reconstruction likelihood of the decoder from our variational distribution; this ensures that the learned distribution is modeling effective latents that the original data can be regenerated from.  The second term measures how similar the learned variational distribution is to a prior belief held over latent variables.  Minimizing this term encourages the encoder to actually learn a distribution rather than collapse into a Dirac delta function.  Maximizing the ELBO is thus equivalent to maximizing its first term and minimizing its second term.

A defining feature of the VAE is how the ELBO is optimized jointly over parameters $\bm{\phi}$ and $\bm{\theta}$.  The encoder of the VAE is commonly chosen to model a multivariate Gaussian with diagonal covariance, and the prior is often selected to be a standard multivariate Gaussian: 
\begin{align}
    q_{\bm{\phi}}(\bm{z}|\bm{x}) &= \mathcal{N}(\bm{z}; \bm{\mu}_{\bm{\phi}}(\bm{x}), \bm{\sigma}_{\bm{\phi}}^2(\bm{x})\textbf{I})\\
    p(\bm{z}) &= \mathcal{N}(\bm{z}; \bm{0}, \textbf{I})
\end{align}
Then, the KL divergence term of the ELBO can be computed analytically, and the reconstruction term can be approximated using a Monte Carlo estimate.  Our objective can then be rewritten as:
\begin{align}
    \scalemath{0.97}{\argmax_{\bm{\phi}, \bm{\theta}} \mathbb{E}_{q_{\bm{\phi}}(\bm{z}|\bm{x})}\left[\log p_{\bm{\theta}}(\bm{x}|\bm{z})\right] - \infdiv{q_{\bm{\phi}}(\bm{z}|\bm{x})}{p(\bm{z})} \approx \argmax_{\bm{\phi}, \bm{\theta}} \sum_{l=1}^{L}\log p_{\bm{\theta}}(\bm{x}|\bm{z}^{(l)}) - \infdiv{q_{\bm{\phi}}(\bm{z}|\bm{x})}{p(\bm{z})}}
\end{align}
where latents $\{\bm{z}^{(l)}\}_{l=1}^L$ are sampled from $q_{\bm{\phi}}(\bm{z}|\bm{x})$, for every observation $\bm{x}$ in the dataset.  However, a problem arises in this default setup: each $\bm{z}^{(l)}$ that our loss is computed on is generated by a stochastic sampling procedure, which is generally non-differentiable.  Fortunately, this can be addressed via the \textit{reparameterization trick} when $q_{\bm{\phi}}(\bm{z}|\bm{x})$ is designed to model certain distributions, including the multivariate Gaussian.

The reparameterization trick rewrites a random variable as a deterministic function of a noise variable; this allows for the optimization of the non-stochastic terms through gradient descent.  For example, samples from a normal distribution $x \sim \mathcal{N}(x;\mu, \sigma^2)$ with arbitrary mean $\mu$ and variance $\sigma^2$ can be rewritten as:
\begin{align*}
    x &= \mu + \sigma\epsilon \quad \text{with } \epsilon \sim \mathcal{N}(\epsilon; 0, \eye)
\end{align*}
In other words, arbitrary Gaussian distributions can be interpreted as standard Gaussians (of which $\epsilon$ is a sample) that have their mean shifted from zero to the target mean $\mu$ by addition, and their variance stretched by the target variance $\sigma^2$.  Therefore, by the reparameterization trick, sampling from an arbitrary Gaussian distribution can be performed by sampling from a standard Gaussian, scaling the result by the target standard deviation, and shifting it by the target mean.

In a VAE, each $\bm{z}$ is thus computed as a deterministic function of input $\bm{x}$ and auxiliary noise variable $\bm{\epsilon}$:
\begin{align*}
    \bm{z} &= \bm{\mu}_{\bm{\phi}}(\bm{x}) + \bm{\sigma}_{\bm{\phi}}(\bm{x})\odot\bm{\epsilon} \quad \text{with } \bm{\epsilon} \sim \mathcal{N}(\bm{\epsilon};\bm{0}, \textbf{I})
\end{align*}
where $\odot$ represents an element-wise product.  Under this reparameterized version of $\bm{z}$, gradients can then be computed with respect to $\bm{\phi}$ as desired, to optimize $\bm{\mu}_{\bm{\phi}}$ and $\bm{\sigma}_{\bm{\phi}}$.  The VAE therefore utilizes the reparameterization trick and Monte Carlo estimates to optimize the ELBO jointly over $\bm{\phi}$ and $\bm{\theta}$.

After training a VAE, generating new data can be performed by sampling directly from the latent space $p(\bm{z})$ and then running it through the decoder.  Variational Autoencoders are particularly interesting when the dimensionality of $\bm{z}$ is less than that of input $\bm{x}$, as we might then be learning compact, useful representations.  Furthermore, when a semantically meaningful latent space is learned, latent vectors can be edited before being passed to the decoder to more precisely control the data generated.

\subsubsection*{Hierarchical Variational Autoencoders}
\addcontentsline{toc}{section}{\protect\numberline{}\protect\numberline{}Hierarchical Variational Autoencoders}%
A Hierarchical Variational Autoencoder (HVAE)~\cite{kingma2016improved, sonderby2016ladder} is a generalization of a VAE that extends to multiple hierarchies over latent variables.  Under this formulation, latent variables themselves are interpreted as generated from other higher-level, more abstract latents. Intuitively, just as we treat our three-dimensional observed objects as generated from a higher-level abstract latent, the people in Plato's cave treat three-dimensional objects as latents that generate their two-dimensional observations.  Therefore, from the perspective of Plato's cave dwellers, their observations can be treated as modeled by a latent hierarchy of depth two (or more).

Whereas in the general HVAE with $T$ hierarchical levels, each latent is allowed to condition on all previous latents, in this work we focus on a special case which we call a Markovian HVAE (MHVAE).  In a MHVAE, the generative process is a Markov chain; that is, each transition down the hierarchy is Markovian, where decoding each latent $\bm{z}_t$ only conditions on previous latent $\bm{z}_{t+1}$.  Intuitively, and visually, this can be seen as simply stacking VAEs on top of each other, as depicted in Figure \ref{fig:hvae}; another appropriate term describing this model is a Recursive VAE.  Mathematically, we represent the joint distribution and the posterior of a Markovian HVAE as:
\begin{align}
    p(\bm{x}, \bm{z}_{1:T}) &= p(\bm{z}_T)p_{\bm{\theta}}(\bm{x}|\bm{z}_1)\prod_{t=2}^{T}p_{\bm{\theta}}(\bm{z}_{t-1}|\bm{z}_{t}) \label{eq:20}\\
    q_{\bm{\phi}}(\bm{z}_{1:T}|\bm{x}) &= q_{\bm{\phi}}(\bm{z}_1|\bm{x})\prod_{t=2}^{T}q_{\bm{\phi}}(\bm{z}_{t}|\bm{z}_{t-1}) \label{eq:21}
\end{align}
Then, we can easily extend the ELBO to be:
\begin{align}
\log p(\bm{x}) &= \log \int p(\bm{x}, \bm{z}_{1:T}) d\bm{z}_{1:T}         && \text{(Apply Equation \ref{eq:1})}\\
&= \log \int \frac{p(\bm{x}, \bm{z}_{1:T})q_{\bm{\phi}}(\bm{z}_{1:T}|\bm{x})}{q_{\bm{\phi}}(\bm{z}_{1:T}|\bm{x})} d\bm{z}_{1:T}         && \text{(Multiply by 1 = $\frac{q_{\bm{\phi}}(\bm{z}_{1:T}|\bm{x})}{q_{\bm{\phi}}(\bm{z}_{1:T}|\bm{x})}$)}\\
&= \log \mathbb{E}_{q_{\bm{\phi}}(\bm{z}_{1:T}|\bm{x})}\left[\frac{p(\bm{x}, \bm{z}_{1:T})}{q_{\bm{\phi}}(\bm{z}_{1:T}|\bm{x})}\right]         && \text{(Definition of Expectation)}\\
&\geq \mathbb{E}_{q_{\bm{\phi}}(\bm{z}_{1:T}|\bm{x})}\left[\log \frac{p(\bm{x}, \bm{z}_{1:T})}{q_{\bm{\phi}}(\bm{z}_{1:T}|\bm{x})}\right]         && \text{(Apply Jensen's Inequality)} \label{eq:25}
\end{align}
\begin{figure}
  \centering
  %includegraphics[width=0.6\linewidth]{images/hvae.png}
  \caption{A Markovian Hierarchical Variational Autoencoder with $T$ hierarchical latents.  The generative process is modeled as a Markov chain, where each latent $\bm{z}_t$ is generated only from the previous latent $\bm{z}_{t+1}$.}
  \label{fig:hvae}
\end{figure}We can then plug our joint distribution (Equation \ref{eq:20}) and posterior (Equation \ref{eq:21}) into Equation \ref{eq:25} to produce an alternate form:
\begin{align}
\mathbb{E}_{q_{\bm{\phi}}(\bm{z}_{1:T}|\bm{x})}\left[\log \frac{p(\bm{x}, \bm{z}_{1:T})}{q_{\bm{\phi}}(\bm{z}_{1:T}|\bm{x})}\right]
&= \mathbb{E}_{q_{\bm{\phi}}(\bm{z}_{1:T}|\bm{x})}\left[\log \frac{p(\bm{z}_T)p_{\bm{\theta}}(\bm{x}|\bm{z}_1)\prod_{t=2}^{T}p_{\bm{\theta}}(\bm{z}_{t-1}|\bm{z}_{t})}{q_{\bm{\phi}}(\bm{z}_1|\bm{x})\prod_{t=2}^{T}q_{\bm{\phi}}(\bm{z}_{t}|\bm{z}_{t-1})}\right]
\end{align}
As we will show below, when we investigate Variational Diffusion Models, this objective can be further decomposed into interpretable components.
\begin{figure}
  \centering
  %includegraphics[width=\linewidth]{images/vdm_base.png}
  \caption{A visual representation of a Variational Diffusion Model; $\bm{x}_0$ represents true data observations such as natural images, $\bm{x}_T$ represents pure Gaussian noise, and $\bm{x}_t$ is an intermediate noisy version of $\bm{x}_0$.  Each $q(\bm{x}_t|\bm{x}_{t-1})$ is modeled as a Gaussian distribution that uses the output of the previous state as its mean.}
  \label{fig:vdm}
\end{figure}

\section{モンテカルロ法}
\subsection{メトロポリス・ヘイスティングス法}


\section{自己学習モンテカルロ法}


\section{自己学習ハイブリッドモンテカルロ法}

% \input{file/review}






\bibliographystyle{unsrt}%参考文bibliographystyle献出力スタイル
\bibliography{myrefs}
\end{document}





